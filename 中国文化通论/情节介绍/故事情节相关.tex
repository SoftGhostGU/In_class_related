\documentclass{article}

% 导入宏包
\usepackage{fancyhdr}
\usepackage{ctex}
\usepackage{listings}
\usepackage{graphicx}
\usepackage[a4paper, body={18cm,22cm}]{geometry}
\usepackage{amsmath,amsthm,amssymb,amstext,wasysym,enumerate,graphicx}
\usepackage{float,abstract,booktabs,indentfirst,amsmath}
\usepackage{array}
\usepackage{multirow}
\usepackage{url}
\usepackage{diagbox}
\usepackage{enumitem}
\usepackage{xcolor}
\usepackage{makecell}
\usepackage{tikz}
\usepackage{tcolorbox}
\usetikzlibrary{positioning, arrows.meta}
\usepackage[bookmarks=true, colorlinks, citecolor=blue, linkcolor=black]{hyperref}


% 设置段落
\renewcommand\arraystretch{1.4}
\setlength{\parindent}{2em}
\setCJKmonofont{黑体}

% 设置高亮文字
\newtcbox{\mybox}[1][red]
{on line, arc = 0pt, outer arc = 0pt,
	colback = #1!10!white, colframe = #1!50!black,
	boxsep = 0pt, left = 1pt, right = 1pt, top = 2pt, bottom = 2pt,
	boxrule = 0pt, bottomrule = 1pt, toprule = 1pt}

% 配置代码显示
\lstset{
	xleftmargin = 3em,
	xrightmargin = 3em,
	aboveskip = 1em,
	backgroundcolor = \color{white},
	basicstyle = \small\ttfamily,
	rulesepcolor = \color{gray},
	breaklines = true,
	numbers = left,
	numberstyle = \small,
	numbersep = -14pt,
	keywordstyle = \color{purple}\bfseries,
	commentstyle = \color{green!60!black}, % 修改注释颜色
	stringstyle = \color{red!60!green!90!blue!90},
	morekeywords = {ASSERT, int64_t, uint32_t},
	moreemph = {ASSERT, NULL},
	emphstyle = \color{red}\bfseries,
	moreemph = [2]{int64\_t, uint32\_t, tid\_t, uint8\_t, int16\_t, uint16\_t, int32\_t, size\_t, bool},
	emphstyle = [2]\color{purple}\bfseries,
	frame = shadowbox,
	showspaces = false,
	columns = fixed
	morecomment = [l][\color{green!60!black}]{+}, % 设置以+开头的代码行为绿色
}

%--------------------页眉--------------------%

\pagestyle{fancy}
\fancyhead[L]{}
\fancyhead[R]{10235101527~顾翌炜}
\fancyhead[C]{中国文化通论期末报告 - 情节介绍}
\fancyfoot[C]{-\thepage-}
\renewcommand{\headrulewidth}{1.5pt}

%--------------------标题--------------------%

\begin{document}
	\begin{center}
		{\Large{\textbf{\heiti 《雪中铃》}}}
	\end{center}
	
	\section{故事基本内容}
	
	\subsection{时间(液态的感知)}
	
	\begin{enumerate}[noitemsep, label={· }]
		\item \textbf{主观时间}: 我的记忆如雨季的溪流时断时续(用不同字体区分记忆碎片)
		\item \textbf{历史锚点}:
		
		\begin{enumerate}[noitemsep, label={· }]
			\item 正德四年霜降(真实瘗旅日期)
			\item 苗历“忌言日”(你施术自噬的那天)
		\end{enumerate}\textbf{}
		
	\end{enumerate}\textbf{}
	
	\subsection{地点(心象化的场景)}
	
	\begin{enumerate}[noitemsep, label={· }]
		\item \textbf{蜈蚣坡}:在我眼中是巨型蜈蚣化石的脊背(每节骨缝里埋着不同时期的你)
		\item \textbf{驿站菜畦}:王阳明种萝卜的地方,我总幻觉下面埋着京城带来的《大明律》
	\end{enumerate}\textbf{}
	
	\subsection{人物(我与他的光影)}
	
	\begin{table}[!ht]
		\centering
		\begin{tabular}{lll}
			\hline
			维度 & 王阳明(我眼中的他) & 阿蕈(我的自我认知)  \\ \hline
			存在形态 & 实体却如镜中倒影 & 虚体却比肉身更真实  \\ 
			语言特征 & 说话时总拂过衣袖(扫去概念尘埃) & 语言碎片化,常混杂苗语古谣  \\ 
			象征物 & 总在修补的破靴(象征“格物”困境) & 会随着记忆增减重量的银项圈  \\ \hline
		\end{tabular}
	\end{table}
	
	\section{故事情节}
	
	\subsection{起因(灵魂的迷雾)}
	
	\textbf{“我醒来时,手里攥着一把带血的草药,却想不起它是救人的白芨,还是杀人的断肠草。”}
	
	\begin{enumerate}[noitemsep, label={· }]
		\item 我是苗疆巫医阿蕈(名字意为“遗忘之草”),因阻止土司焚烧汉人医书被咒术反噬,记忆破碎,只记得破碎的画面:
		
			\begin{enumerate}[noitemsep, label={· }]
				\item 一个穿官服的男人在暴雨中掩埋尸体
				\item 铜铃上“良知”二字在火中闪烁
				\item 有人对你说:“你葬的不是死人,是未醒的活人”
			\end{enumerate}\textbf{}
	\end{enumerate}\textbf{}
	
	\subsection{经过(镜像中的觉醒)}
	
	\begin{enumerate}[noitemsep, label={· }]
		\item 第一阶段:误认
		
		\begin{enumerate}[noitemsep, label={· }]
			\item 我在蜈蚣坡撞见王阳明埋葬吏目,他蓑衣上的冰碴让你莫名流泪
			\item 我断言:“这三人生前中了记忆蛊,需开棺取脑施术”——他按住我的手腕:\textbf{“小伙子,他们只是冻死的凡人。”}
		\end{enumerate}\textbf{}
		
		\item 第二阶段:共瘗
		
		\begin{enumerate}[noitemsep, label={· }]
			\item 他教我用汉礼挖坟时,我的手指突然自行摆出苗疆祭祀手印(身体记忆苏醒)
			\item 夜里我发热呓语,他守着火堆轻念:\textbf{“心若不安,遍地皆是龙场;心安处,瘴雾亦作青云。”}
		\end{enumerate}\textbf{}
		
		\item 第三阶段:裂变
		
		\begin{enumerate}[noitemsep, label={· }]
			\item 我发现那具“吏目尸体”腰间的玉佩,竟和我梦中反复出现的图案一致
			\item 王阳明将玉佩对着月光:“你看,裂纹里长出蕨草了——死境中有生机,正如迷途中有觉路。”
		\end{enumerate}\textbf{}
		
	\end{enumerate}\textbf{}
	
	\subsection{结果(永恒的未完成)}
	
	\begin{enumerate}[noitemsep, label={· }]
		\item 最终我没有恢复全部记忆,但决定留在龙场
		
		\begin{enumerate}[noitemsep, label={· }]
			\item 清晨:我教苗童用草药汁临摹《瘗旅文》
			\item 黄昏:王阳明在你的铜铃上补刻“知行”二字
		\end{enumerate}\textbf{}
		
		\item 最后一个场景
		
		\begin{enumerate}[noitemsep, label={· }]
			\item \textbf{“雪落在当年埋尸的山坡上,我和他成了彼此的记忆容器。”}
		\end{enumerate}\textbf{}
		
	\end{enumerate}\textbf{}
	
	\section{意象系统解析}
	
	\subsection{核心意象群}
	
	\begin{table}[H]
		\centering
		\begin{tabular}{lll}
			\hline
			意象 & 物质属性 & 象征意义 \\ \hline
			铜铃 & \makecell[l]{锈蚀的苗疆法器,刻"良知"二字;\\阿蕈记忆复苏的关键触发器} & 被遮蔽的本心/汉苗文化碰撞的伤口   \\ 
			黑雨 & \makecell[l]{含铁锈味的酸性雨水;\\每次记忆闪回时的环境背景} & 历史暴力(廷杖/贬谪)的液态记忆   \\ 
			蕨类植物 & \makecell[l]{从尸体裂缝萌发的嫩芽;\\坟头、玉佩裂纹、铜铃内部} & 绝境中的顿悟/心学"致良知"的生命力   \\ \hline
			
		\end{tabular}
	\end{table}
	
	\begin{table}[H]
		\centering
		\begin{tabular}{lll}
			\hline
			意象 & 物质属性 & 象征意义  \\ \hline
			银项圈 & \makecell[l]{随记忆增减重量的苗银首饰;\\阿蕈身份认知变化的标尺} & 文化认同的度量衡/被篡改的历史重量   \\ 
			瘴气 & \makecell[l]{乳白色致幻雾气;\\王阳明讲笑话时的环境反讽} & 理学教条的窒息感/未被思想照亮的精神迷雾   \\ \hline
		\end{tabular}
	\end{table}
	
	\subsection{身体意象群}
	
	\begin{table}[!ht]
		\centering
		\begin{tabular}{lll}
			\hline
			意象 & 物质表现 & 哲学映射  \\ \hline
			腕间伤疤 & 枷锁磨出的紫红色陈旧伤 & 体制暴力在思想者肉体上的铭刻  \\ 
			咳出的血沫 & 肺疾导致的粉红色泡沫 & 思想如同内出血,在体内自我消化而后绽放  \\ 
			童子尿 & 温热的淡黄色液体浸透衣袍 & 圣贤之道从"高高在上"到"俯就卑微"的隐喻  \\ \hline
		\end{tabular}
	\end{table}
	
	\subsection{生活器物意象群}
	
	\begin{table}[!ht]
		\centering
		\begin{tabular}{lll}
			\hline
			意象 & 历史原型 & 心学解构  \\ \hline
			破蓑衣 & 龙场驿丞官服替代品 & 剥离社会身份后的本真存在状态  \\ 
			糙米粥 & 掺野芹毒菇的生存食物 & "格物"的终极形态——将苦难熬煮成滋养  \\ 
			竹枝笔 & 折断后用来书写的天然工具 & 思想不依赖精致载体(朱熹式穷理→阳明式顿悟) \\ \hline
		\end{tabular}
	\end{table}
	
	\subsection{超现实意象群}
	
	\begin{table}[!ht]
		\centering
		\begin{tabular}{lll}
			\hline
			意象 & 魔幻表现 & 现实对应  \\ \hline
			青铜树 & 会结冰爆裂的祭祀神树 & 程朱理学体系僵化结构的象征  \\ 
			记忆蛊 & 吞噬特定记忆的透明蠕虫 & 权力对历史叙事的系统性抹除  \\ 
			血铃铛 & 锈蚀到极致时渗出人血的铜铃 & 思想传承中的疼痛性分娩 \\ \hline
		\end{tabular}
	\end{table}
	
	\clearpage
	
	\subsection{附录}
	
	写作前期使用白话文方式写了这篇文章,但是考虑到故事主人公是古代人物,用文言写作能更自然地唤醒读者对那个时代的文化记忆,让故事更有历史真实感,就像用古琴演奏古曲,比用钢琴更对味。故本文最终版采用文言文来展示。
	
	故此处将两个版本的查看链接放置于此(点击可以访问):
	
	\begin{enumerate}[noitemsep, label={{\arabic*})}]
		\item \textbf{白话文版}:\href{https://github.com/SoftGhostGU/In_class_related/blob/main/%E4%B8%AD%E5%9B%BD%E6%96%87%E5%8C%96%E9%80%9A%E8%AE%BA/%E3%80%8A%E9%9B%AA%E4%B8%AD%E9%93%83%E3%80%8B-%E7%99%BD%E8%AF%9D%E6%96%87%E7%89%88/%E3%80%8A%E9%9B%AA%E4%B8%AD%E9%93%83%E3%80%8B-%E5%8E%9F%E6%96%87%E7%89%88.pdf}{原文(白话文版)链接}
		\item \textbf{文言文版}:\href{https://github.com/SoftGhostGU/In_class_related/blob/main/%E4%B8%AD%E5%9B%BD%E6%96%87%E5%8C%96%E9%80%9A%E8%AE%BA/%E3%80%8A%E9%9B%AA%E4%B8%AD%E9%93%83%E3%80%8B/%E3%80%8A%E9%9B%AA%E4%B8%AD%E9%93%83%E3%80%8B.pdf}{文言文版链接}(该版本为提交版本)
	\end{enumerate}\textbf{}
	
\end{document}
