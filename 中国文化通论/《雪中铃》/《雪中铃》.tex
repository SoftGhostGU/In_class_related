\documentclass{article}

% 导入宏包
\usepackage{fancyhdr}
\usepackage{ctex}
\usepackage{listings}
\usepackage{graphicx}
\usepackage[a4paper, body={18cm,22cm}]{geometry}
\usepackage{amsmath,amsthm,amssymb,amstext,wasysym,enumerate,graphicx}
\usepackage{float,abstract,booktabs,indentfirst,amsmath}
\usepackage{array}
\usepackage{multirow}
\usepackage{url}
\usepackage{diagbox}
\usepackage{enumitem}
\usepackage{xcolor}
\usepackage{makecell}
\usepackage{tikz}
\usepackage{tcolorbox}
\usetikzlibrary{positioning, arrows.meta}
\usepackage[bookmarks=true, colorlinks, citecolor=blue, linkcolor=black]{hyperref}
\usepackage{titlesec}
\usepackage{setspace}
\usepackage[utf8]{inputenc}

% 设置section标题格式,不显示章节数字
\titleformat{\section}[hang]{\normalfont\Large\bfseries}{}{0em}{}[]
% 设置subsection标题格式,不显示章节数字
\titleformat{\subsection}[hang]{\normalfont\Large\bfseries}{}{0em}{}[\titlerule]


% 设置段落
\renewcommand\arraystretch{1.4}
\setlength{\parindent}{2em}
\setCJKmonofont{黑体}

% 设置高亮文字
\newtcbox{\mybox}[1][red]
{on line, arc = 0pt, outer arc = 0pt,
	colback = #1!10!white, colframe = #1!50!black,
	boxsep = 0pt, left = 1pt, right = 1pt, top = 2pt, bottom = 2pt,
	boxrule = 0pt, bottomrule = 1pt, toprule = 1pt}

% 配置代码显示
\lstset{
	xleftmargin = 3em,
	xrightmargin = 3em,
	aboveskip = 1em,
	backgroundcolor = \color{white},
	basicstyle = \small\ttfamily,
	rulesepcolor = \color{gray},
	breaklines = true,
	numbers = left,
	numberstyle = \small,
	numbersep = -14pt,
	keywordstyle = \color{purple}\bfseries,
	commentstyle = \color{green!60!black}, % 修改注释颜色
	stringstyle = \color{red!60!green!90!blue!90},
	morekeywords = {ASSERT, int64_t, uint32_t},
	moreemph = {ASSERT, NULL},
	emphstyle = \color{red}\bfseries,
	moreemph = [2]{int64\_t, uint32\_t, tid\_t, uint8\_t, int16\_t, uint16\_t, int32\_t, size\_t, bool},
	emphstyle = [2]\color{purple}\bfseries,
	frame = shadowbox,
	showspaces = false,
	columns = fixed
	morecomment = [l][\color{green!60!black}]{+}, % 设置以+开头的代码行为绿色
}

% 1.5倍行间距
\linespread{1.5}

% 段落间距
\setlength{\parskip}{6pt}

%--------------------页眉--------------------%

\pagestyle{fancy}
\fancyhead[L]{}
\fancyhead[R]{10235101527~顾翌炜}
\fancyhead[C]{中国文化通论期末报告}
\fancyfoot[C]{-\thepage-}
\renewcommand{\headrulewidth}{1.5pt}

%--------------------标题--------------------%

\begin{document}
	\begin{center}
		{\Large{\textbf{\heiti 雪中铃}}}
	\end{center}
	
	\subsection{第一章:黑雨启}
	
	黑雨初歇,予醒于蜈蚣坡北。断柯横涧,瘴雾吞峰,四野阒寂若太古。腐殖之气先醒,挟千年败叶叠葬复掘之息,自指隙渗入骨髓。十指陷淤,状若蕨根吮地脉,苍白如死,而暗汲黄泉之忆。
	
	腰间铜铃忽颤,其声浊若衔枚。审视之,见“良知”二字蚀于铜锈,裂纹纵横若涸涧。指抚其文,忽忆幼时师言:“铜铁易锈,心印难蚀。”今铃尚在,而铸者谁耶?
	
	予竟不识己为谁。
	
	然坡下三坟新垒,土犹腥湿。一耆老、一童仆、一默者,皆吾手瘗——甲缝凝霜土,掌横血槽可证。顾彼容貌死因,乃至瘗之故,尽付渺茫。岂知铜铃即心印,锈蚀如我渡忘川?
	
	雾中忽现蓑衣人,敝服下现枷痕若兽啮。其掘土时,腕骨嶙峋似剑,忽仰目相视,瞳中映残炬:“汝至矣。”语方出,雨骤歇,而青铜古树冰裂声彻谷。
	
	“葬人者自葬,觉否?”其人掷锸而笑,声若碎玉,“昔朱熹格竹,今予格尸,皆求死理耳。”
	
	予大恸,指坟茔欲语,忽见掌心血槽竟生蕨芽。雾散处,残阳如血,照见三碑无字,唯一碑阴刻苗文:“葬昨者,永囿今。”
	
	是时顿悟:非为葬人,实觅己尸耳。而此身朽否?铃忽绽,蕨种飞焰中,现诏狱铁窗雪景——三年前,原是我亲手将此铃掷入囚车。
	
	\subsection{第二章:雾中的名字}
	
	雨初歇,湿气凝而未散,蜈蚣坡上青灰之雾渐起,如游魂徘徊于林隙。枯枝垂露,坠地无声,而腐土之气隐隐浮动,似有无名之怨蕴于其中。
	
	彼男子自号“驿丞”者,立于新冢之侧,形容瘦削,眉目间似有阴翳。其袖中探出粗麻布一方,色如陈年骨殖,乃徐徐拭其铁锹。刃上泥垢虽去,而血色暗沉,竟不知是新沾雨露,抑或旧染猩红。其指节嶙峋,动作极缓,若恐惊扰冢中长眠之客。余观之,忽忆及苗疆老巫拭祭器之态——彼辈素言:“物之灵久于人魂,器若蒙尘,则怨气凝结。”思及此,背脊陡生寒意。
	
	余以足尖轻点坟土,腐殖之下竟传空洞回响,如幽泉呜咽。“识得此中亡者否?”余问。
	
	驿丞摇首,其腕间旧伤横亘,于晨光中泛紫,似毒蛇啮痕,又似枷锁磨迹。“然知其死状。”其声沙哑,如枯叶摩挲碑石。
	
	“何以死?”余追问。
	
	“未携雨具耳。”
	
	初闻似戏言,然察其双目,静若深潭止水,寒气森然。
	
	\subsection{第三章:铜铃的饥饿}
	
	铜铃复作鸣响,其声不复腰间震颤,竟自喉间渗出,如鲠在喉。锈蚀铜绿若活物般循血脉游走,所过之处,肌肤泛起青灰之色。余每欲言语,辄闻铜铃轻颤,似有万千细足在皮下爬行,啃噬记忆如蠹鱼蚀简。
	
	驿丞蹲踞篝火旁,煮一釜所谓粥食。野芹枯黄,糙米陈腐,于沸水中翻腾起浊沫,散发出霉湿之气。其手持断竹为箸,搅动时竹节刮擦釜底,其声如钝刀刮骨,令人毛骨悚然。火光映照下,其面容阴晴不定,眸中似有鬼火明灭。
	
	“汝之铃,”彼忽开口,声若枯枝折断,“乃旧年故物。”
	
	余急按喉间铜铃,顿觉铜锈腥气在口中炸裂,舌根泛起铁锈滋味。“荒谬!”
	
	“三载前,”驿丞舀起一勺浊粥,热气蒸腾如冤魂吐息,“有苗女夜盗太医院,适逢缇骑押解死囚。”其腕间旧伤在火光下泛着紫黑之色,细观之,分明是铁链经年累月磨出的沟壑,深可见骨。
	
	“那囚犯......”
	
	“廷杖三记,肋骨尽折。”彼将粥碗递来,浊汤中浮沉着几茎枯芹,“较之诏狱馊食,此犹胜三分。”
	
	余凝视铜铃,见其表面蚩尤纹路间渗出血丝。驿丞腕伤忽渗黑血,滴落尘土竟化作数只赤蚁,疾走而去。夜风过处,远处传来铁链拖曳之声,似有若无。
	
	\subsection{第四章:坟前的蕨}
	
	三日,新冢之上,忽见蕨芽破土而出。其色青嫩,蜷曲如婴儿之拳,叶脉间犹带夜露,映晨光而微微颤动。驿丞跪坐坟前,以指甲掐断一茎,碧色汁液自断处渗出,沾染其指,竟隐隐泛着幽光,似有灵性。
	
	“汝可信人死化草之说?”彼忽开口,声音低沉,如自地底传来。
	
	余忆及苗疆故老所言,谓枉死之人,魂灵不散,往往化为食肉之藤,缠绕过客。“不信。”余答。
	
	“吾则深信。”驿丞将蕨芽含入口中,细细咀嚼,唇齿间绿汁蜿蜒而下,“昔在诏狱,曾见一书生受杖而亡。其血渗入青砖缝隙,翌日,砖缝间竟生毒蕈,其色如血。”
	
	言未毕,余忽觉一阵眩晕,腰间铜铃无风自动,剧烈震颤。细观之,锈蚀裂纹间竟渗出暗红液体,黏稠似凝血。驿丞抬首视余,嘴角犹带蕨汁之绿:“今可忆起?汝所葬者,非他人也。”
	
	余心神俱震,蓦然醒悟:“乃吾自身。”
	
	暮色渐沉,铜铃之声渐止,而冢上蕨芽愈盛,新叶舒展,似欲攫人。驿丞指尖绿痕未消,而余喉间铜锈之气愈浓,竟与坟土腥气混作一处,再难分辨。
	
	\subsection{第五章:黑雨再临}
	
	日暮时分,天光骤暗。雨落如丝,其质黏腻,触肤若蛛网缠身,寒意浸骨。驿丞独立坡顶,蓑衣翻飞,露出内里衣衫——初看似官服褪色,细观之,实乃血迹斑驳之囚衣,暗红如锈。
	
	“初见汝时,”其声混于雨声,飘忽不定,“便知是索债而来。”
	
	铜铃忽挣脱束缚,飞入其掌。雨洗斑驳锈迹,露出底下铭文:“知善知恶是良知”,笔划如刀,寒光凛冽。
	
	雷光乍现,照见冢前墓碑——无名无姓,唯有苗文咒语深刻:“埋葬昨日者,永世不得解脱”,字迹蜿蜒如蛇。
	
	雨幕渐密,驿丞身形淡若轻烟。而吾颈间银项圈骤沉,重若千钧,似有无数亡魂附坠其上。雨声呜咽,铜铃寂然,唯有咒文在碑上隐隐泛光,与项圈相映,皆为囚心之枷。
	
	\subsection{第六章:瘴气与笑话}
	
	龙场之瘴,晨时最盛。乳白雾气如冤魂素帛,缠绕口鼻,令人气息凝滞。驿丞独于此时说笑,其声穿雾而来,嘶哑若虫鸣。
	
	“知黔地蚯蚓何以独长乎?”彼蹲踞菜畦,以指掘湿土,现出粉红蠕虫一截,“盖欲穿山越岭,赴京告于陛下:‘律法至此,尚不若一虫耳。’”
	
	本应发笑,然喉间忽涌铁锈之气。视腰间铜铃,其纹愈裂,暗红斑驳若凝血痂。
	
	“无趣耶?”驿丞仰首,蓑衣泥渍斑斑,“另说一典——昔有痴人被贬蛮荒,见此地虫豸,竟比圣贤书更通天道。”
	
	余骤攫湿土掷其面:“病骨支离,何故作笑?”
	
	泥土自其颊滑落,彼安然若素:“盖因咳嗽之声,较之跪拜,更近天地呼吸。”
	
	瘴雾弥散,蚯蚓入土不见。铜铃微颤,其声暗哑,似应和远方雷鸣。驿丞咳声断续,竟与地脉颤动同频。湿土气息、铜锈腥气、药渣苦气,皆混入晨雾,再难分辨。
	
	\subsection{第七章:童尿与圣贤}
	
	晨起,童子溺裤,年方龆龀,裤管淋漓,面赪若朱。驿丞见状拊掌而笑,遽解青袍裹之,曰:“善哉!童子溺乃辟瘴良方,尔此溺可活阖驿性命。”其声朗朗,穿林越壑。
	
	及夜,残月窥窗,见其挑灯补袍。银针穿梭,线迹蜿蜒若百足虫行。余问:“待众人皆以赤心,虽溺床竖子亦不忍呵?”
	
	驿丞啮断丝缕,目注灯花:“知其溺裤之由否?”
	
	“岂惧山魈邪?”
	
	“惧步冢中三人后尘耳。”其指摩挲补处,布纹间犹渗溺痕,“伊父为驿卒,去岁驰递公文,冻毙风雪,半躯为豺所噬。”
	
	铜铃骤喑,万籁俱寂。惟闻漏滴声声,竟较往昔铮鸣更催人肝肠。
	
	\subsection{第八章:最后的粥}
	
	隆冬时节,朔风怒号,大雪封山。驿丞于破败灶台前烹煮最后一釜粥。釜中粟米稀疏可数,米粒在沸水中上下沉浮。彼却从容投以野芹一束,毒蕈两朵(沸久则毒性尽消),又添自鼠穴夺来野栗三颗。釜中物事翻滚不息,浊浪排空,白雾升腾,将破败驿站笼罩其中。
	
	“此中道理,岂非暗合朱子格物之说?”其执半截焦黑木箸徐徐搅动,嘴角噙着若有若无的笑意,“天地造化,终归一釜糊涂。”
	
	余注视其皲裂十指,见甲缝间渗着暗红血丝:“若吾...实乃奉命取汝性命之人?”
	
	“当趁吾格竹穷理之时下手。”其舀起半勺粥羹轻吹,白气氤氲间眉目模糊,“彼时神游物外,纵死亦不失为糊涂鬼。”
	
	窗外风雪愈急,檐下青铜树渐结冰凌,铮铮然如冰弦骤断,其声凄厉。积雪压枝,时有枝桠断裂之声传来。
	
	“请君品鉴。”其将粗陶碗推至案前,碗沿犹带裂痕,“此粥入腹,前尘往事,当如明镜。”
	
	\subsection{第九章:铜铃花开}
	
	铜铃轰然迸裂,其声穿云裂石。斑驳铜锈簌簌剥落,内里银光乍现,皎若秋霜。细观之,铃中竟无铜舌,唯存一粒干瘪蕨实,此刻忽绽新芽,嫩叶舒展如碧玉雕琢。
	
	王守仁——不复驿丞之貌——独立风雪之中,唇边朱红点点。其身着囚衣已褪尽本色,化作蝉翼般的空壳,在朔风中飘摇不定,似欲乘风归去。
	
	“原来非是苗疆巫医。”他以指拭血,指尖在苍白的唇上留下一道殷红,“竟是当年刑部门前,为死囚送水的那个小药童。”
	
	记忆如决堤之水奔涌而至。三载之前,锦衣卫押解囚车经过刑部衙门,一垂髫童子冲破人群,将铜铃掷入槛车之中。那铃铛滚落血污,其声暗哑。
	
	“大人!”我双膝没入积雪,手中银铃忽生异变,无数蕨类孢子如烟似雾,在寒风中流转,“可还记得这枚铃铛?”
	
	守仁闻言大笑,笑声未竟便化作剧咳,身形佝偻如弓。咳声震落枯枝积雪,惊起寒鸦数只:“如今...咳咳...该刻'行'字了......既已'知'之...咳咳...花开满枝......”
	
	\subsection{第十章:未完成的埋葬}
	
	吾终未言吏目怀中密信所载之事,亦未道破那“冻毙少年”实乃刘瑾暗遣之刺客。此二事如鲠在喉,终化作雪中一声叹息。
	
	积雪渐厚,没至膝处。指尖触及他腕间旧伤,皮下骨茬突起如刃,触之生寒。此乃三年前廷杖所断之骨,至今未愈,每逢阴雨便隐隐作痛。
	
	“汝知乎?”其忽掬雪覆面,寒霜染眉,须发皆白。“昔在诏狱,鼠辈夜夜啮吾伤口时,吾竟顿悟圣贤真谛。”
	
	“愿闻其详。”吾握紧铜铃,锈屑簌簌而落。
	
	“所谓圣贤之道,”雪水顺颊而下,混入血色,滴落雪地如红梅绽放,“不过教人骨肉遭噬之际,犹能含笑为鼠辈歌一曲《阳春》。”
	
	铜铃忽作清鸣,其声穿雪而来,至此方得明晰。余音袅袅,与远处狼嚎相应和。
	
	\clearpage
	
	\subsection{附录}
	
	考虑到故事主人公是古代人物,用文言写作能更自然地唤醒读者对那个时代的文化记忆,让故事更有历史真实感,就像用古琴演奏古曲,比用钢琴更对味。故本文采用文言文来展示。
	
	我在写作前期写过一份\textbf{情节与小说意象介绍}与\textbf{本文的白话文版本},有助于我的写作。当然也可以帮助阅读,故此处将查看链接放置于此:
	
	(点击可以访问)
	
	\begin{enumerate}[noitemsep, label={{\arabic*})}]
		\item \textbf{情节\&意象介绍}:\href{https://github.com/SoftGhostGU/In_class_related/blob/main/%E4%B8%AD%E5%9B%BD%E6%96%87%E5%8C%96%E9%80%9A%E8%AE%BA/%E6%83%85%E8%8A%82%E4%BB%8B%E7%BB%8D/%E6%95%85%E4%BA%8B%E6%83%85%E8%8A%82%E7%9B%B8%E5%85%B3.pdf}{情节\&意象介绍链接}
		\item \textbf{白话文版}:\href{https://github.com/SoftGhostGU/In_class_related/blob/main/%E4%B8%AD%E5%9B%BD%E6%96%87%E5%8C%96%E9%80%9A%E8%AE%BA/%E3%80%8A%E9%9B%AA%E4%B8%AD%E9%93%83%E3%80%8B-%E7%99%BD%E8%AF%9D%E6%96%87%E7%89%88/%E3%80%8A%E9%9B%AA%E4%B8%AD%E9%93%83%E3%80%8B-%E5%8E%9F%E6%96%87%E7%89%88.pdf}{原文(白话文版)链接}
	\end{enumerate}\textbf{}
	
\end{document}
