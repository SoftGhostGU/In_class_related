\documentclass{article}

% 导入宏包
\usepackage{fancyhdr}
\usepackage{ctex}
\usepackage{listings}
\usepackage{graphicx}
\usepackage[a4paper, body={18cm,22cm}]{geometry}
\usepackage{amsmath,amsthm,amssymb,amstext,wasysym,enumerate,graphicx}
\usepackage{float,abstract,booktabs,indentfirst,amsmath}
\usepackage{array}
\usepackage{multirow}
\usepackage{url}
\usepackage{diagbox}
\usepackage{enumitem}
\usepackage{xcolor}
\usepackage{makecell}
\usepackage{tikz}
\usepackage{tcolorbox}
\usetikzlibrary{positioning, arrows.meta}
\usepackage[bookmarks=true, colorlinks, citecolor=blue, linkcolor=black]{hyperref}
\usepackage{titlesec}
\usepackage{setspace}
\usepackage[utf8]{inputenc}

% 设置section标题格式,不显示章节数字
\titleformat{\section}[hang]{\normalfont\Large\bfseries}{}{0em}{}[]
% 设置subsection标题格式,不显示章节数字
\titleformat{\subsection}[hang]{\normalfont\Large\bfseries}{}{0em}{}[\titlerule]


% 设置段落
\renewcommand\arraystretch{1.4}
\setlength{\parindent}{2em}
\setCJKmonofont{黑体}

% 设置高亮文字
\newtcbox{\mybox}[1][red]
{on line, arc = 0pt, outer arc = 0pt,
	colback = #1!10!white, colframe = #1!50!black,
	boxsep = 0pt, left = 1pt, right = 1pt, top = 2pt, bottom = 2pt,
	boxrule = 0pt, bottomrule = 1pt, toprule = 1pt}

% 配置代码显示
\lstset{
	xleftmargin = 3em,
	xrightmargin = 3em,
	aboveskip = 1em,
	backgroundcolor = \color{white},
	basicstyle = \small\ttfamily,
	rulesepcolor = \color{gray},
	breaklines = true,
	numbers = left,
	numberstyle = \small,
	numbersep = -14pt,
	keywordstyle = \color{purple}\bfseries,
	commentstyle = \color{green!60!black}, % 修改注释颜色
	stringstyle = \color{red!60!green!90!blue!90},
	morekeywords = {ASSERT, int64_t, uint32_t},
	moreemph = {ASSERT, NULL},
	emphstyle = \color{red}\bfseries,
	moreemph = [2]{int64\_t, uint32\_t, tid\_t, uint8\_t, int16\_t, uint16\_t, int32\_t, size\_t, bool},
	emphstyle = [2]\color{purple}\bfseries,
	frame = shadowbox,
	showspaces = false,
	columns = fixed
	morecomment = [l][\color{green!60!black}]{+}, % 设置以+开头的代码行为绿色
}

% 1.5倍行间距
\linespread{1.5}

% 段落间距
\setlength{\parskip}{6pt}


%--------------------页眉--------------------%

\pagestyle{fancy}
\fancyhead[L]{}
\fancyhead[R]{10235101527~顾翌炜}
\fancyhead[C]{中国文化通论期末报告}
\fancyfoot[C]{-\thepage-}
\renewcommand{\headrulewidth}{1.5pt}

%--------------------标题--------------------%

\begin{document}
	
	\begin{center}
		{\Large{\textbf{\heiti 雪中铃(白话文版)}}}
	\end{center}
	
	\subsection{第一章:黑雨启}
	
	\textbf{我是在一场黑雨中醒来的。}
	
	泥土的气味最先回到我身体里——潮湿的、腐殖的,带着被落叶反复掩埋又翻出的陈旧。我的手指陷在泥中,像某种植物的根须,苍白而静默地汲取着大地的记忆。  
	
	铜铃在腰间轻轻响动,声音闷闷的,像是被什么堵住了喉咙。我低头看它,铜锈斑驳的表面刻着两个字:“良……知……” 笔画断裂处爬满细小的裂纹,像一条干涸的河。  
	
	我不知道自己是谁。  
	
	但我知道,这片山坡上埋着三个人。  
	
	一个老人,一个少年,一个沉默的仆人。他们的尸体是我亲手掩埋的——至少,我的身体记得这件事。我的指甲缝里还残留着冻土的碎屑,掌心横贯着铁锹磨出的血痕。可我不记得他们的脸,不记得他们为何而死,甚至不记得自己为何要埋葬他们。  
	
	我只记得一个画面:
	
	一个瘦削的男人站在雨里,蓑衣下的官服早已褪色。他弯腰挖土时,袖口滑落,露出手腕上一道狰狞的旧伤——像是被什么野兽撕咬过,又像是被铁链磨出的烙印。  
	
	他抬头看我,眼睛里映着将熄的火把。  
	
	“你来了。”
	
	他说这句话时,雨突然停了。  
	
	而我终于想起——  
	
	\textbf{我并不是来埋葬他们的。 }
	
	\textbf{我是来寻找自己的尸体的。} 
	
	\subsection{第二章:雾中的名字}
	
	\textbf{雨停后,蜈蚣坡升起青灰色的雾。}  
	
	那个男人——他让我叫他“驿丞”,可他的眼睛分明在说这不是真名——从袖中取出一块粗麻布,慢条斯理地擦拭铁锹上的泥。他的动作太细致了,像是在对待什么圣物。我忽然想起苗寨里那些擦拭祭器的老巫师,他们总说器物比人记得更久。  
	
	“你认识他们?”我踢了踢脚下的新坟,腐殖土发出空洞的回响。  
	
	他摇头,腕间的伤疤在晨光中泛着淡紫色:“但我知道他们怎么死的。”  
	
	“怎么死的?”  
	
	“死于忘记带伞。”他说。  
	
	\textbf{我以为他在说笑,可他的眼神却像冻住的溪水,平静得让人发冷。  }
	
	
	\subsection{第三章:铜铃的饥饿}
	
	\textbf{铜铃又开始响了。 } 
	
	这次它不是在腰间震动,而是在我的喉咙里。我能感觉到那些锈蚀的铜锈正顺着我的血管爬行,像某种寄生的藤蔓,啃食着我残存的记忆。  
	
	驿丞在煮粥。说是粥,其实不过是把野芹和糙米丢进沸水里,任它们翻滚出浑浊的泡沫。他蹲在火堆旁,用一根折断的竹枝搅动铁锅,竹节刮擦锅底的声响让我牙酸。  
	
	“你的铃,”他突然说,“以前是我的。”  
	
	我猛地按住铜铃,铜锈的腥气在舌尖炸开。  
	
	“不可能。”  
	
	“三年前,有个苗女在京城太医院偷药书。”他舀起一勺粥,吹散热气,“她撞见了锦衣卫押送囚犯,慌乱中掉落了这铃铛。”  
	
	我盯着他手腕上的伤疤——现在我看清了,那不是野兽的咬痕,是铁链磨出的血肉模糊。  
	
	“那个囚犯……”  
	
	“吃了三记廷杖,断了三根肋骨。”他递来粥碗,“喝吗?比诏狱的馊饭强些。”  
	
	
	\subsection{第四章:坟前的蕨}
	
	\textbf{第三天,坟头长出了蕨。  }
	
	嫩绿的卷芽从土缝里钻出,像婴儿攥紧的拳头。驿丞跪坐在坟前,用指甲掐断一株蕨芽,汁液染绿了他的指尖。  
	
	“你相信人死后会变成植物吗?”他问。  
	
	我想到苗疆的传说:枉死之人会化作食肉的藤,缠绕过路的活物。  
	
	“不信。”  
	
	“我信。”他把蕨芽含进嘴里,慢慢咀嚼,“在诏狱时,我见过一个书生被杖毙。他的血渗进砖缝,第二天,砖缝里长出了蘑菇。”  
	
	我忽然感到一阵眩晕。铜铃在腰间剧烈震颤,那些锈蚀的裂纹中渗出暗红的液体,像凝固的血。  
	
	驿丞抬头看我,嘴角还沾着蕨汁的绿色:“想起来了吗?你埋葬的不是他们。”  
	
	\textbf{“是我自己。” } 
	
	
	\subsection{第五章:黑雨再临}
	
	\textbf{傍晚时,天又黑了。 } 
	
	这一次的雨比记忆中的更粘稠,落在皮肤上像冰冷的蛛网。驿丞站在坡顶,蓑衣被风掀起,露出里面褪色的官服——原来那根本不是官服,而是一件血迹斑斑的囚衣。  
	
	“第一次见到你时,”他的声音混在雨声里,“我就知道你是来讨债的。”  
	
	铜铃终于挣脱束缚,飞向他的掌心。那些斑驳的锈迹在雨中剥落,露出底下完整的刻字:  
	
	“知善知恶是良知。”
	
	雷声炸响的瞬间,我终于看清了坟前的墓碑——那上面没有名字,只有我用苗文刻下的咒语:  
	
	“埋葬昨日者,永世不得解脱。”
	
	\textbf{雨幕中,驿丞的身影渐渐透明。 } 
	
	\textbf{而我的银项圈,终于重若千钧。 }
	
	
	\subsection{第六章:瘴气与笑话}
	
	\textbf{龙场的瘴气在清晨最浓,乳白的雾像冤魂的裹尸布,缠住人的口鼻。驿丞却在这时讲起了笑话。}
	
	“你知道贵州的蚯蚓为什么特别长吗?”他蹲在菜畦里,手指挖开潮湿的泥土,露出一截粉红色的蠕虫,“因为它们要钻过整座山,去告诉京城的皇帝——‘陛下,您的律法在这里,还不如一条虫’。”  
	
	我本该笑的,可喉咙里却涌上一股铁锈味。铜铃的裂痕更深了,那些暗红的锈斑像干涸的血痂。  
	
	“不好笑?”他抬头看我,蓑衣上沾满泥点,“那换个故事——有个傻子被贬到蛮荒之地,发现这里的虫子比圣贤书更懂天道。”  
	
	我忽然抓起一把土摔向他:“你明明快病死了,为什么还在笑?”  
	
	他任由泥土从脸上滑落:\textbf{“因为我发现,咳嗽比跪拜更接近天地。”  }
	
	\subsection{第七章:童尿与圣贤}
	
	\textbf{童子尿了裤子。 } 
	
	那小鬼不过八九岁,裤管滴滴答答漏着水,脸涨得比山里的毒蘑菇还红。驿丞却大笑起来,脱下自己的外袍裹住他:“妙哉!童子尿乃辟瘴良药,你这是要救我们全驿站的命啊!”  
	
	夜里,我见他借着月光补那件尿湿的袍子。针脚歪歪扭扭,像蜈蚣爬过的痕迹。  
	
	“你对谁都这么好吗?”我问,“连个尿床的小厮都舍不得骂?”  
	
	他咬断线头:“你知道他为什么尿裤子吗?”  
	
	“怕鬼?”  
	
	“怕自己变成那三个坟里的人。”他摩挲着补丁,“他爹是驿站马夫,去年冻死在送公文路上,尸体被狼啃了一半。”  
	
	\textbf{铜铃突然安静了。这一刻,我宁愿它继续响。} 
	
	\subsection{第八章:最后的粥}
	
	\textbf{雪落下来的那天,驿丞在煮最后一锅粥。}  
	
	锅里的米少得能数清,他却扔进一把野芹、两朵毒蘑菇(“煮过就没毒了”)、三颗从鼠洞里抢来的野栗。  
	
	“像不像朱熹的‘格物’?”他搅着锅笑,“把天地万物都炖成一锅糊涂。”  
	
	我盯着他开裂的指甲:“如果……我是来杀你的呢?”  
	
	“那你该趁我格竹子的时候动手。”他舀起一勺粥吹气,“那时候我满脑子都是‘竹子的道理’,死了也是糊涂鬼。”  
	
	\textbf{雪越下越大。我们头顶的青铜树开始结冰,发出琴弦崩断般的脆响。 } 
	
	“喝吧。”他把粥碗推给我,“喝完这碗,你就能想起自己是谁了。” 
	
	\subsection{第九章:铜铃花开}
	
	我是在铜铃的爆裂声中恢复记忆的。  
	
	\textbf{那些锈蚀的铜皮一片片剥落,露出内里崭新的白银。铃铛里没有铜舌,只有一粒干枯的蕨种,此刻正抽出嫩绿的芽。}  
	
	驿丞——不,王守仁——站在雪地里咳血。他的囚衣终于完全褪色,变成一张半透明的皮,像蛇蜕下的旧梦。  
	
	“原来你不是巫医。”他擦掉嘴角的血,“是那个在刑部门口,给囚犯喂水的药童。”  
	
	记忆如黑雨倾泻而下。三年前,锦衣卫押解他出京时,有个小药童冲出人群,把铜铃塞进囚车。  
	
	“大人!”我跪在雪地里,手中银铃开满蕨类细小的孢子,“您还认得这个铃吗?”  
	
	他笑得咳弯了腰:“现在它该刻‘行’字了……咳咳……毕竟‘知’已经开花了。” 
	
	\subsection{第十章:最终章:未完成的埋葬}
	
	\textbf{我没有告诉他,吏目怀里那封密信写着什么。}  
	
	\textbf{也没有说,那个“冻死的少年”其实是刘瑾派来的刺客。}  
	
	雪埋住我们膝盖时,我摸到了他腕间伤疤下的硬块——那是廷杖打断的骨头,至今未愈的棱角。  
	
	“你知道吗?”他抓起一把雪按在脸上,“当年在诏狱,老鼠啃我的伤口时,我突然明白了——”  
	
	“明白什么?”  
	
	“原来圣贤之道,不过是教人如何在被啃噬时,还能给老鼠唱支歌。”  
	
	\textbf{铜铃最后响了一次。 } 
	
	\textbf{这次,我们都听清了它的声音。}
	
	\clearpage
	
	\subsection{附录}
	
	我在写作前期写过一份\textbf{情节与小说意象介绍},有助于我的写作。当然也可以帮助阅读。
	
	同时,考虑到故事主人公是古代人物,用文言写作能更自然地唤醒读者对那个时代的文化记忆,让故事更有历史真实感,就像用古琴演奏古曲,比用钢琴更对味。故本次期末作业最终版采用文言文来展示。
	
	故此处将查看链接放置于此(点击可以访问):
	
	\begin{enumerate}[noitemsep, label={{\arabic*})}]
		\item \textbf{情节\&意象介绍}:\href{https://github.com/SoftGhostGU/In_class_related/blob/main/%E4%B8%AD%E5%9B%BD%E6%96%87%E5%8C%96%E9%80%9A%E8%AE%BA/%E6%83%85%E8%8A%82%E4%BB%8B%E7%BB%8D/%E6%95%85%E4%BA%8B%E6%83%85%E8%8A%82%E7%9B%B8%E5%85%B3.pdf}{情节\&意象介绍链接}
		\item \textbf{文言文版}:\href{https://github.com/SoftGhostGU/In_class_related/blob/main/%E4%B8%AD%E5%9B%BD%E6%96%87%E5%8C%96%E9%80%9A%E8%AE%BA/%E3%80%8A%E9%9B%AA%E4%B8%AD%E9%93%83%E3%80%8B/%E3%80%8A%E9%9B%AA%E4%B8%AD%E9%93%83%E3%80%8B.pdf}{文言文版链接}(该版本为提交版本)
	\end{enumerate}\textbf{}
	
\end{document}
