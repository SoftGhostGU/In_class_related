\documentclass{article}

% 导入宏包
\usepackage{fancyhdr}
\usepackage{ctex}
\usepackage{listings}
\usepackage{graphicx}
\usepackage[a4paper, body={18cm,22cm}]{geometry}
\usepackage{amsmath,amsthm,amssymb,amstext,wasysym,enumerate,graphicx}
\usepackage{float,abstract,booktabs,indentfirst,amsmath}
\usepackage{array}
\usepackage{multirow}
\usepackage{url}
\usepackage{diagbox}
\usepackage{enumitem}
\usepackage{xcolor}
\usepackage{makecell}
\usepackage{tikz}
\usepackage{tcolorbox}
\usetikzlibrary{positioning, arrows.meta}
\usepackage[bookmarks=true, colorlinks, citecolor=blue, linkcolor=black]{hyperref}


% 设置段落
\renewcommand\arraystretch{1.4}
\setlength{\parindent}{2em}
\setCJKmonofont{黑体}

% 设置高亮文字
\newtcbox{\mybox}[1][red]
{on line, arc = 0pt, outer arc = 0pt,
	colback = #1!10!white, colframe = #1!50!black,
	boxsep = 0pt, left = 1pt, right = 1pt, top = 2pt, bottom = 2pt,
	boxrule = 0pt, bottomrule = 1pt, toprule = 1pt}

% 配置代码显示
\lstset{
	xleftmargin = 3em,
	xrightmargin = 3em,
	aboveskip = 1em,
	backgroundcolor = \color{white},
	basicstyle = \small\ttfamily,
	rulesepcolor = \color{gray},
	breaklines = true,
	numbers = left,
	numberstyle = \small,
	numbersep = -14pt,
	keywordstyle = \color{purple}\bfseries,
	commentstyle = \color{green!60!black}, % 修改注释颜色
	stringstyle = \color{red!60!green!90!blue!90},
	morekeywords = {ASSERT, int64_t, uint32_t},
	moreemph = {ASSERT, NULL},
	emphstyle = \color{red}\bfseries,
	moreemph = [2]{int64\_t, uint32\_t, tid\_t, uint8\_t, int16\_t, uint16\_t, int32\_t, size\_t, bool},
	emphstyle = [2]\color{purple}\bfseries,
	frame = shadowbox,
	showspaces = false,
	columns = fixed
	morecomment = [l][\color{green!60!black}]{+}, % 设置以+开头的代码行为绿色
}

%--------------------页眉--------------------%

\pagestyle{fancy}
\fancyhead[L]{}
\fancyhead[R]{}
\fancyhead[C]{华东师范大学软件工程学院作业}
\fancyfoot[C]{-\thepage-}
\renewcommand{\headrulewidth}{1.5pt}

%--------------------标题--------------------%

\begin{document}
	
	\begin{center}
		{\Large{\textbf{\heiti 软件工程学院数据库系统及其应用作业}}}
		\begin{table}[htb]
			\flushleft
			\begin{tabular}{p{0.4\linewidth}p{0.27\linewidth}p{0.28\linewidth}}\\
				\textbf{实验课程}:数据库系统及其应用  & \textbf{年级}:2023级       & \textbf{姓名}:顾翌炜  \\
				\textbf{作业编号}:Week-2    & \textbf{学号}:10235101527 & \textbf{作业日期}:2025/02/26  \\
			\end{tabular}
		\end{table}
	\end{center}
	\rule{\textwidth}{2pt}
	
	\setlength{\parindent}{2em}
	
    \section*{2.6}
    
    请考虑图2-17的职员数据库。请给出关系代数表达式来表示下面的每个查询: 
    \begin{enumerate}
    	\item 请找出居住在城市“Miami”的每位职员的姓名。
    	\item 请找出薪水超过 \$100000的每位职员的姓名。
    	\item 请找出居住在“Miami”并且薪水超过 \$100000的每位职员的姓名。
    \end{enumerate}
    
    \begin{tcolorbox}[title = {2-17的职员数据库}, colback = blue!25!white, colframe = blue!75!black]
    	employee (ID, person\_name,street, city)\\
    	works (ID, company\_name, salary) \\
    	company(company\_name, city)
    \end{tcolorbox}
    
    \section*{2.6解答}
    
    \begin{enumerate}[noitemsep, label={{\arabic*})}]
    	\item $\pi_{person\_name}(\sigma_{city='Miami'}(employee))$
    	\item $\pi_{person\_name}(\sigma_{salary>100000}(employee \bowtie works))$
    	\item $\pi_{person\_name}(\sigma_{city='Miami'}(employee) \bowtie \sigma_{salary>100000}(works))$
    \end{enumerate}\textbf{}
    
    \section*{2.7}
    
    请考虑图 2-18的银行数据库。请给出关系代数表达式来表示下面的每个查询:
    \begin{enumerate}
    	\item 请找出位于“Chicago”的每家支行的名称。
    	\item 请找出在“Downtown”支行有贷款的每位贷款人的 ID。
    \end{enumerate}
    
    \begin{tcolorbox}[title = {2-18的银行数据库}, colback = blue!25!white, colframe = blue!75!black]
    	branch(branch name, branch.city, assets)\\
    	customer (ID, customer.name,customer street,customercity) \\
    	loan (loan number, branch.name, amount) \\
    	borrower (ID, loan\_number)\\
    	account (accountnumber, branchname, balance) depositor (ID, account number)
    \end{tcolorbox}
    
    \section*{2.7解答}
    
    \begin{enumerate}[noitemsep, label={{\arabic*})}]
    	\item $\pi_{branch\_name}(\sigma_{branch\_city = 'Chicago'}(branch))$
    	\item $\pi_{ID} (\sigma_{branch.name = 'Downtown'}(loan \bowtie borrower))$
    \end{enumerate}\textbf{}
	
	\section*{2.8}
	
	请考虑图 2-17的职员数据库。请给出关系代数表达式来表示下面的每个查询:
	
	\begin{enumerate}
		\item 请找出不为“BigBank”工作的每位职员的 ID 和姓名。
		\item 请找出至少与数据库中每位职员的薪水同样多(薪资最高)的所有职员的ID 和姓名。
	\end{enumerate}
	
	\begin{tcolorbox}[title = {2-17的职员数据库}, colback = blue!25!white, colframe = blue!75!black]
		employee (ID, person\_name,street, city)\\
		works (ID, company\_name, salary) \\
		company(company\_name, city)
	\end{tcolorbox}
	
	\section*{2.8解答}
	
	\begin{enumerate}[noitemsep, label={{\arabic*})}]
		\item $\pi_{ID, person\_name} (employee) - \pi_{ID, person\_name} (\sigma_{company\_name = 'BigBank'} (employee \bowtie works))$
		
		\item $\pi_{ID, person\_name} ( \sigma_{salary = max\_salary} (employee \bowtie works) )$ 
		
		其中:$max\_salary = \gamma_{MAX(salary)} (works)$
		
	\end{enumerate}\textbf{}
	
\end{document}
