\documentclass{article}

% 导入宏包
\usepackage{fancyhdr}
\usepackage{ctex}
\usepackage{listings}
\usepackage{graphicx}
\usepackage[a4paper, body={18cm,22cm}]{geometry}
\usepackage{amsmath,amsthm,amssymb,amstext,wasysym,enumerate,graphicx}
\usepackage{float,abstract,booktabs,indentfirst,amsmath}
\usepackage{array}
\usepackage{multirow}
\usepackage{url}
\usepackage{diagbox}
\usepackage{enumitem}
\usepackage{xcolor}
\usepackage{makecell}
\usepackage{tikz}
\usepackage{tcolorbox}
\usetikzlibrary{positioning, arrows.meta}
\usepackage[bookmarks=true, colorlinks, citecolor=blue, linkcolor=black]{hyperref}


% 设置段落
\renewcommand\arraystretch{1.4}
\setlength{\parindent}{2em}
\setCJKmonofont{黑体}

% 设置高亮文字
\newtcbox{\mybox}[1][red]
{on line, arc = 0pt, outer arc = 0pt,
	colback = #1!10!white, colframe = #1!50!black,
	boxsep = 0pt, left = 1pt, right = 1pt, top = 2pt, bottom = 2pt,
	boxrule = 0pt, bottomrule = 1pt, toprule = 1pt}

% 配置代码显示
\lstset{
	xleftmargin = 3em,
	xrightmargin = 3em,
	aboveskip = 1em,
	backgroundcolor = \color{white},
	basicstyle = \small\ttfamily,
	rulesepcolor = \color{gray},
	breaklines = true,
	numbers = left,
	numberstyle = \small,
	numbersep = -14pt,
	keywordstyle = \color{purple}\bfseries,
	commentstyle = \color{green!60!black}, % 修改注释颜色
	stringstyle = \color{red!60!green!90!blue!90},
	morekeywords = {ASSERT, int64_t, uint32_t},
	moreemph = {ASSERT, NULL},
	emphstyle = \color{red}\bfseries,
	moreemph = [2]{int64\_t, uint32\_t, tid\_t, uint8\_t, int16\_t, uint16\_t, int32\_t, size\_t, bool},
	emphstyle = [2]\color{purple}\bfseries,
	frame = shadowbox,
	showspaces = false,
	columns = fixed
	morecomment = [l][\color{green!60!black}]{+}, % 设置以+开头的代码行为绿色
}

%--------------------页眉--------------------%

\pagestyle{fancy}
\fancyhead[L]{}
\fancyhead[R]{}
\fancyhead[C]{华东师范大学软件工程学院作业}
\fancyfoot[C]{-\thepage-}
\renewcommand{\headrulewidth}{1.5pt}

%--------------------标题--------------------%

\begin{document}
	
	\begin{center}
		{\Large{\textbf{\heiti 软件工程学院数据库系统及其应用作业}}}
		\begin{table}[htb]
			\flushleft
			\begin{tabular}{p{0.4\linewidth}p{0.27\linewidth}p{0.28\linewidth}}\\
				\textbf{实验课程}:数据库系统及其应用  & \textbf{年级}:2023级       & \textbf{姓名}:顾翌炜  \\
				\textbf{作业编号}:Week-3    & \textbf{学号}:10235101527 & \textbf{作业日期}:2025/03/05  \\
			\end{tabular}
		\end{table}
	\end{center}
	\rule{\textwidth}{2pt}
	
	\setlength{\parindent}{2em}
	
	\section*{1}
	
	使用本讲课件中的 \textbf{university} 示例,完成如下查询:
	
	\begin{tcolorbox}[title = {课件中的university}, colback = blue!25!white, colframe = blue!75!black]
		$instructor(ID, name, dept\_name, salary)$ \\
		$student(ID, name, dept\_name, tot\_cred)$ \\
		$takes(ID, course\_id, sec\_id, semester, year, grade)$ \\
		$sourse(course\_id, title, dept\_name, credits)$
	\end{tcolorbox}
	
	\begin{enumerate}
		\item[a)] 至少选修了一门 \textbf{Comp. Sci.} 课程的每名学生的 ID 和姓名,保证结果中没有重复的姓名;
		
		\item[b)] 找出没有选修 2017 年之前开设的任何课程的每名学生的 ID 和姓名;
		
		\item[c)] 找出每个系的教师的最高工资值;
		
		\item[d)] 从前述查询所计算出的每个系的最高工资中选出所以系中的最低值。
	\end{enumerate}
	
	\section*{1 解答}
	
	\begin{enumerate}[noitemsep, label={{\arabic*})}]
		\item[a)] 
		
		\begin{verbatim}
			select distinct s.ID, s.name
			from student s, takes t, course c
			where s.ID = t.ID
			      and t.course_id = c.course_id
			      and c.dept_name = 'Comp.Sci';
		\end{verbatim}
		
		\item[b)]
		
		\begin{verbatim}
			select distinct s.ID, s.name
			from student s
			where s.ID not in (
			      select t.ID
			      from takes t
			      where t.year < 2017
			);
		\end{verbatim}
		
		\item[c)]
		
		\begin{verbatim}
			select dept_name, max(salary) as max_salary
			from instructor
			group by dept_name;
		\end{verbatim}
		
		\item[d)] 
		
		\begin{verbatim}
			select min(max_salary)
			from (select dept_name, max(salary) as max_salary
			      from instructor
			      group by dept_name) as max_salary
		\end{verbatim}
	\end{enumerate}\textbf{}
	
	\section*{2}
	
	\noindent
	
	\begin{tcolorbox}[title = {雇员数据库}, colback = blue!25!white, colframe = blue!75!black]
		$employee(ID, person\_name, street, city)$ \\
		$works(ID, company\_name, salary)$ \\
		$company(company\_name, city)$ \\
		$manages(ID, manager\_id)$
	\end{tcolorbox}
	
	\noindent
	
	\textbf{查询:}
	
	\begin{enumerate}
		\item[a).] 为 `First Copr` 的所有雇员增长 10\% 的工资;
		\item[b).] 删除 `First Copr` 的雇员在 works 关系中的所有元组。
	\end{enumerate}
	
	\section*{2 解答}
	
	\begin{enumerate}[noitemsep, label={{\arabic*})}]
		\item[a)] 
		
		\begin{verbatim}
			update works
			set salary = salary * 1.1
			where company_name = 'First Copr'
		\end{verbatim}
		
		\item[b)]
		
		\begin{verbatim}
			delete from works
			where company_name = 'First Copr'
		\end{verbatim}
		
	\end{enumerate}\textbf{}
	
\end{document}
