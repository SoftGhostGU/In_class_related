\documentclass{article}

% 导入宏包
\usepackage{fancyhdr}
\usepackage{ctex}
\usepackage{listings}
\usepackage{graphicx}
\usepackage[a4paper, body={18cm,22cm}]{geometry}
\usepackage{amsmath,amsthm,amssymb,amstext,wasysym,enumerate,graphicx}
\usepackage{float,abstract,booktabs,indentfirst,amsmath}
\usepackage{array}
\usepackage{multirow}
\usepackage{url}
\usepackage{diagbox}
\usepackage{enumitem}
\usepackage{xcolor}
\usepackage{makecell}
\usepackage{tikz}
\usepackage{tcolorbox}
\usetikzlibrary{positioning, arrows.meta}
\usepackage[bookmarks=true, colorlinks, citecolor=blue, linkcolor=black]{hyperref}


% 设置段落
\renewcommand\arraystretch{1.4}
\setlength{\parindent}{2em}
\setCJKmonofont{黑体}

% 设置高亮文字
\newtcbox{\mybox}[1][red]
{on line, arc = 0pt, outer arc = 0pt,
	colback = #1!10!white, colframe = #1!50!black,
	boxsep = 0pt, left = 1pt, right = 1pt, top = 2pt, bottom = 2pt,
	boxrule = 0pt, bottomrule = 1pt, toprule = 1pt}

% 配置代码显示
\lstset{
	xleftmargin = 3em,
	xrightmargin = 3em,
	aboveskip = 1em,
	backgroundcolor = \color{white},
	basicstyle = \small\ttfamily,
	rulesepcolor = \color{gray},
	breaklines = true,
	numbers = left,
	numberstyle = \small,
	numbersep = -14pt,
	keywordstyle = \color{purple}\bfseries,
	commentstyle = \color{green!60!black}, % 修改注释颜色
	stringstyle = \color{red!60!green!90!blue!90},
	morekeywords = {ASSERT, int64_t, uint32_t},
	moreemph = {ASSERT, NULL},
	emphstyle = \color{red}\bfseries,
	moreemph = [2]{int64\_t, uint32\_t, tid\_t, uint8\_t, int16\_t, uint16\_t, int32\_t, size\_t, bool},
	emphstyle = [2]\color{purple}\bfseries,
	frame = shadowbox,
	showspaces = false,
	columns = fixed
	morecomment = [l][\color{green!60!black}]{+}, % 设置以+开头的代码行为绿色
}

%--------------------页眉--------------------%

\pagestyle{fancy}
\fancyhead[L]{}
\fancyhead[R]{}
\fancyhead[C]{华东师范大学软件工程学院作业}
\fancyfoot[C]{-\thepage-}
\renewcommand{\headrulewidth}{1.5pt}

%--------------------标题--------------------%

\begin{document}
	
	\begin{center}
		{\Large{\textbf{\heiti 软件工程学院数据库系统及其应用作业}}}
		\begin{table}[htb]
			\flushleft
			\begin{tabular}{p{0.4\linewidth}p{0.27\linewidth}p{0.28\linewidth}}\\
				\textbf{实验课程}:数据库系统及其应用  & \textbf{年级}:2023级       & \textbf{姓名}:顾翌炜  \\
				\textbf{作业编号}:Week-5    & \textbf{学号}:10235101527 & \textbf{作业日期}:2025/03/20  \\
			\end{tabular}
		\end{table}
	\end{center}
	\rule{\textwidth}{2pt}
	
	\setlength{\parindent}{2em}
	
	\section*{5.7}
	
	Consider the bank database of Figure 5.21. Write an SQL trigger to carry out the following action: On \textbf{delete} of an account, for each customer-owner of the account, check if the owner has any remaining accounts, and if she does not, delete her from the \textit{depositor} relation.
	
	\begin{lstlisting}[language=, title=Banking Database, tabsize=4]
		branch (branch_name, branch_city, assets)
		customer (customer_name, customer_street, customer_city)
		loan (loan_number, branch_name, amount)
		borrower (customer_name, loan_number)
		account (account_number, branch_name, balance)
		depositor (customer_name, account_number)
	\end{lstlisting}
	
	\section*{5.7 解答}
	
	\begin{verbatim}
		CREATE TRIGGER delete_check
		AFTER DELETE ON account
		  REFERENCING old row as orow
		  FOR EACH ROW
		  BEGIN
		    DELETE FROM depositor
		    WHERE customer_name NOT IN (SELECT d1.customer_name
		                                FROM depositor d1
		                                WHERE d1.account_number <> orow.account_number)
		   END
	\end{verbatim}
	
	\section*{5.8}
	
	Given a relation $ S(\textbf{student}, \textbf{subject}, \textbf{marks}) $, write a query to find the top 10 students by total marks, by using SQL ranking. Include all students tied for the final spot in the ranking, even if that results in more than 10 total students.
	
	\section*{5.8 解答}
	
	\begin{verbatim}
		SELECT *
		FROM (SELECT student, total_marks, RANK() OVER (ORDER BY total_marks DESC) AS s_rank
		      FROM (SELECT student, sum(marks) AS total_marks
		            FROM S
		            GROUP BY student)
		     )
		WHERE s_rank <= 10
	\end{verbatim}
	
	\section*{5.10}
	
    Using the relation from Exercise 5.9, write an SQL query to generate a report showing the number of shares traded, number of trades, and total dollar volume broken down by year, each month of each year, and each trading day.
    
    \textbf{Relation for 5.10:}
    $
    $nyse $ (\textbf{year}, \textbf{month}, \textbf{day}, $shares\_traded$, $dollar\_volume$, $num\_trades$)
    $
    
    \textit{['nyse' means: New York Stock Exchange]}
	
	\section*{5.10 解答}
	
	\begin{verbatim}
		SELECT year, month, day, 
		  sum(shares_traded) as sum_share, 
		  sum(num_trades) as num_trades, 
		  sum(dollar_volume) as total_dollar
		FROM nyse
		GROUP BY ROLLUP (year, month, day)
	\end{verbatim}
\end{document}
