\documentclass{article}

% 导入宏包
\usepackage{fancyhdr}
\usepackage{ctex}
\usepackage{listings}
\usepackage{graphicx}
\usepackage[a4paper, body={18cm,22cm}]{geometry}
\usepackage{amsmath,amsthm,amssymb,amstext,wasysym,enumerate,graphicx}
\usepackage{float,abstract,booktabs,indentfirst,amsmath}
\usepackage{array}
\usepackage{multirow}
\usepackage{url}
\usepackage{diagbox}
\usepackage{enumitem}
\usepackage{xcolor}
\usepackage{makecell}
\usepackage{tikz}
\usepackage{tcolorbox}
\usetikzlibrary{positioning, arrows.meta}
\usepackage[bookmarks=true, colorlinks, citecolor=blue, linkcolor=black]{hyperref}


% 设置段落
\renewcommand\arraystretch{1.4}
\setlength{\parindent}{2em}
\setCJKmonofont{黑体}

% 设置高亮文字
\newtcbox{\mybox}[1][red]
{on line, arc = 0pt, outer arc = 0pt,
	colback = #1!10!white, colframe = #1!50!black,
	boxsep = 0pt, left = 1pt, right = 1pt, top = 2pt, bottom = 2pt,
	boxrule = 0pt, bottomrule = 1pt, toprule = 1pt}

% 配置代码显示
\lstset{
	xleftmargin = 3em,
	xrightmargin = 3em,
	aboveskip = 1em,
	backgroundcolor = \color{white},
	basicstyle = \small\ttfamily,
	rulesepcolor = \color{gray},
	breaklines = true,
	numbers = left,
	numberstyle = \small,
	numbersep = -14pt,
	keywordstyle = \color{purple}\bfseries,
	commentstyle = \color{green!60!black}, % 修改注释颜色
	stringstyle = \color{red!60!green!90!blue!90},
	morekeywords = {ASSERT, int64_t, uint32_t},
	moreemph = {ASSERT, NULL},
	emphstyle = \color{red}\bfseries,
	moreemph = [2]{int64\_t, uint32\_t, tid\_t, uint8\_t, int16\_t, uint16\_t, int32\_t, size\_t, bool},
	emphstyle = [2]\color{purple}\bfseries,
	frame = shadowbox,
	showspaces = false,
	columns = fixed
	morecomment = [l][\color{green!60!black}]{+}, % 设置以+开头的代码行为绿色
}

%--------------------页眉--------------------%

\pagestyle{fancy}
\fancyhead[L]{}
\fancyhead[R]{}
\fancyhead[C]{华东师范大学软件工程学院作业}
\fancyfoot[C]{-\thepage-}
\renewcommand{\headrulewidth}{1.5pt}

%--------------------标题--------------------%

\begin{document}
	
	\begin{center}
		{\Large{\textbf{\heiti 软件工程学院数据库系统及其应用作业}}}
		\begin{table}[htb]
			\flushleft
			\begin{tabular}{p{0.4\linewidth}p{0.27\linewidth}p{0.28\linewidth}}\\
				\textbf{实验课程}:数据库系统及其应用  & \textbf{年级}:2023级       & \textbf{姓名}:顾翌炜  \\
				\textbf{作业编号}:Week-14-15    & \textbf{学号}:10235101527 & \textbf{作业日期}:2025/06/11  \\
			\end{tabular}
		\end{table}
	\end{center}
	\rule{\textwidth}{2pt}
	
	\setlength{\parindent}{2em}
	
	\section*{1 冲突可串行化与优先图 (Conflict Serializability)}
	
	\textbf{背景知识}:一个调度(Schedule)是冲突可串行化的,当且仅当它的优先图(Precedence Graph)是无环的 。优先图中,节点代表事务,如果事务 Ti 的一个操作与 Tj 的一个操作冲突,并且 Ti 的操作先于 Tj 执行,则在图中画一条从 Ti 指向 Tj 的边 。
	
	\textbf{问题}:给定以下包含事务 T1 和 T2 的调度 S
	
	\begin{table}[!ht]
		\centering
		\begin{tabular}{|l|l|l|}
			\hline
			时间 & T1 & T2  \\ \hline
			1 & read(A) &   \\ \hline
			2 & A = A - 100 &   \\ \hline
			3 & ~ & read(A)  \\ \hline
			4 & ~ & A = A * 1.1  \\ \hline
			5 & ~ & write(A)  \\ \hline
			6 & write(A) &   \\ \hline
			7 & read(B) &   \\ \hline
			8 & ~ & read(B)  \\ \hline
			9 & ~ & B = B + 100  \\ \hline
			10 & ~ & write(B)  \\ \hline
			11 & B = B - 10 &   \\ \hline
			12 & write(B) &   \\ \hline
			13 & commit &   \\ \hline
			14 & ~ & commit  \\ \hline
		\end{tabular}
	\end{table}
	
	\textbf{要求}:
	
	\begin{enumerate}[noitemsep, label={{\arabic*})}]
		\item 找出调度 S 中所有的冲突操作对。
		
		\item 画出该调度 S 的优先图(Precedence Graph)。
		
		\item 判断该调度 S 是否是冲突可串行化的,并解释你的结论。如果可串行化,请给出一个与之等价的串行调度。
	\end{enumerate}\textbf{}
	
	\section*{1 解答}
	
	\textbf{冲突操作对}
	
	冲突操作的定义:两个操作属于不同事务,作用于同一数据项,且至少有一个是写操作(write)。
	
	\begin{table}[!ht]
		\centering
		\begin{tabular}{|l|l|}
			\hline
			冲突操作对 & 冲突原因  \\ \hline
			(T1: read(A), T2: write(A)) & T1 读 A 后,T2 写 A(时间顺序:T1(1) → T2(5))  \\ \hline
			(T1: write(A), T2: read(A)) & T2 读 A 后,T1 写 A(时间顺序:T2(3) → T1(6))  \\ \hline
			(T1: write(A), T2: write(A)) & T1 和 T2 都对 A 写(时间顺序:T2(5) → T1(6))  \\ \hline
			(T1: read(B), T2: write(B)) & T2 读 B 后,T1 写 B(时间顺序:T2(7) → T1(10))  \\ \hline
			(T1: write(B), T2: read(B)) & T2 写 B 后,T1 读 B(时间顺序:T2(8) → T1(12))  \\ \hline
			(T1: write(B), T2: write(B)) & T2 和 T1 都对 B 写(时间顺序:T2(10) → T1(12))  \\ \hline
		\end{tabular}
	\end{table}
	
	\begin{tikzpicture}[
		node distance=3cm,
		transaction/.style={circle, draw, minimum size=1cm, font=\large}
		]
		% 定义节点(事务)
		\node[transaction] (T1) {T1};
		\node[transaction, right of=T1] (T2) {T2};
		
		% 绘制双向边(表示环)
		\draw[-{Stealth[scale=1.2]}, thick] (T1) to [bend left=20] node[above] {\small \texttt{}} (T2);
		\draw[-{Stealth[scale=1.2]}, thick] (T2) to [bend left=20] node[below] {\small \texttt{}} (T1);
		
		% 添加标题
		\node[above=0.5cm of T1, font=\bfseries] {优先图(Precedence Graph)};
	\end{tikzpicture}
	
	\textbf{是否可串行化}
	
	由于优先图里存在环T1 → T2 → T1,所以不可串行化
	
	\section*{2 严格两阶段锁协议 (Strict 2PL)}
	
	\textbf{背景知识}: 两阶段锁协议 (2PL) 分为增长阶段(只加锁,不解锁)和缩减阶段(只解锁,不加锁)。严格两阶段锁协议 (Strict 2PL) 要求事务必须持有其所有排他锁 (Exclusive Lock)直到事务提交或中止 ,这可以避免级联回滚 。大多数数据库实现的都是严格两阶段锁协议 。
	
	\textbf{问题}: 假设数据库系统采用严格两阶段锁协议 (Strict 2PL),并且锁管理器使用以下锁相容性矩阵:
	
	\begin{table}[!ht]
		\centering
		\begin{tabular}{|l|l|l|}
			\hline
			(T1 持有) & S (共享锁) & X (排他锁)  \\ \hline
			(T2 请求) S & 兼容 (Yes) & 不兼容 (No)  \\ \hline
			(T2 请求) X & 不兼容 (No) & 不兼容 (No)  \\ \hline
		\end{tabular}
	\end{table}
	
	现有两个事务 T1 和 T2:
	
	\begin{enumerate}[noitemsep, label={{\arabic*})}]
		\item T1: read(A), write(A), commit
		\item T2: read(A), read(B), write(B), commit
	\end{enumerate}\textbf{}
	
	\textbf{要求}: 描述当 T1 和 T2 按以下顺序交叉执行时,锁的申请(lock-S/lock-X)、授予(granted)、拒绝/等待(denied/wait)和释放(unlock)的过程。请清晰地列出每一步操作和锁的状态。
	
	\begin{table}[H]
		\centering
		\begin{tabular}{|l|l|l|l|l|}
			\hline
			时间 & 操作 & T1 的动作 & T2 的动作 & 锁管理器状态  \\ \hline
			1 & T1 & read(A) & ~ &   \\ \hline
			2 & T2 & ~ & read(A) &   \\ \hline
			3 & T1 & write(A) & ~ &   \\ \hline
			4 & T2 & ~ & read(B) &   \\ \hline
			5 & T2 & ~ & write(B) &   \\ \hline
			6 & T1 & commit & ~ &   \\ \hline
			7 & T2 & ~ & commit &   \\ \hline
		\end{tabular}
	\end{table}
	
	
	\section*{2 解答}
	
	\begin{table}[H]
		\centering
		\begin{tabular}{|l|l|l|l|l|l|}
			\hline
			时间 & 操作 & T1 的动作 & T2 的动作 & 锁管理器状态 & 说明  \\ \hline
			1 & T1 & read(A) & ~ & A: S(T1) & T1 获取 A 的 S 锁。 \\ \hline
			2 & T2 & ~ & read(A) & A: S(T1), S(T2) & S 锁兼容,T2 也获取 A 的 S 锁。 \\ \hline
			3 & T1 & write(A) & ~ & A: S(T2), X(T1) 等待 & \makecell[l]{T1 需升级为 X 锁,但 T2 持有 S 锁,\\不兼容。T1 阻塞,等待 T2 释放 S 锁}。 \\ \hline
			4 & T2 & ~ & read(B) &  A: S(T2), X(T1) 等待;B: S(T2) & T2 获取 B 的 S 锁。 \\ \hline
			5 & T2 & ~ & write(B) & A: S(T2), X(T1) 等待;B: X(T2) & T2 升级 B 的锁为 X 锁。 \\ \hline
			6 & T1 & commit & ~ & A: S(T2), X(T1) 释放;B: X(T2) & T1 提交后仅释放自己的 S(A) \\ \hline
			7 & T2 & ~ & commit & A: 无锁;B: 无锁(全释放) & T2 提交后释放所有锁 \\ \hline
		\end{tabular}
	\end{table}
	
	\section*{3 使用日志进行故障恢复 (Recovery with Log)}
	
	\textbf{背景知识}: 现代数据库普遍采用预写日志(Write-Ahead Logging, WAL)策略进行故障恢复。 当系统崩溃后,恢复管理器通过分析(Analysis)、重做(Redo)和 撤销(Undo)三个阶段来恢复数据 。
	
	\begin{enumerate}[noitemsep, label={{\arabic*})}]
		\item Redo 阶段: 重放所有已提交事务以及未完成事务的更新,以确保已提交的修改被持久化 。
		\item Undo 阶段: 回滚所有在崩溃时仍处于活动状态(未提交或未中止)的事务,以保证事务的原子性 
	\end{enumerate}\textbf{}
	
	\textbf{问题}: 给定以下一份简化的数据库日志,系统在日志末尾处发生崩溃。
	
	\begin{table}[!ht]
		\centering
		\begin{tabular}{|l|l|l|}
			\hline
			LSN & 日志记录 & 描述  \\ \hline
			01 & <T1, BEGIN> & T1 开始  \\ \hline
			02 & <T2, BEGIN> & T2 开始  \\ \hline
			03 & <T1, A, 10, 20> & T1 将 A 从 10 改为 20  \\ \hline
			04 & <T3, BEGIN> & T3 开始  \\ \hline
			05 & <T2, B, 50, 60> & T2 将 B 从 50 改为 60  \\ \hline
			06 & <T1, COMMIT> & T1 提交  \\ \hline
			07 & <CHECKPOINT> & 检查点  \\ \hline
			08 & <T2, C, 80, 90> & T2 将 C 从 80 改为 90  \\ \hline
			09 & <T3, D, 40, 30> & T3 将 D 从 40 改为 30  \\ \hline
			10 & <T2, COMMIT> & T2 提交  \\ \hline
			11 & <T3, A, 20, 10> & T3 将 A 从 20 改为 10  \\ \hline
			~ & <-- 系统崩溃 -- &   \\ \hline
		\end{tabular}
	\end{table}
	
	\textbf{要求}: 根据上述日志,描述在系统重启恢复期间:
	
	\begin{enumerate}[noitemsep, label={{\arabic*})}]
		\item 哪些事务需要被 Undo?
		\item 哪些日志记录 (根据 LSN) 需要被 Redo?
		\item 简要说明为什么 T1 事务不需要任何操作。
	\end{enumerate}\textbf{}
	
	\section*{3 解答}
	
	根据日志记录和崩溃时间点(LSN 11 之后),各事务的状态如下:
	
	\begin{enumerate}[noitemsep, label={{\arabic*})}]
		\item T1: 已提交(LSN 06)。
		\item T2: 已提交(LSN 10)。
		\item T3: 未提交(无 COMMIT 记录,最后操作在 LSN 11)。
	\end{enumerate}\textbf{}
	
	\textbf{UNDO和REDO}
	
	\begin{enumerate}[noitemsep, label={{\arabic*})}]
		\item \textbf{UNDO}:
		
		所以需要被undo的事务是:T3唯一在崩溃时仍活跃(未提交)的事务,其修改必须回滚
		
		首先undo LSN 11(<T3, A, 20, 10>):将 A 从 10 改回 20。
		
		再undo LSN 09(<T3, D, 40, 30>):将 D 从 30 改回 40。
		
		\item \textbf{REDO}:
		
		T1已经提交,T3需要undo,只有T2需要redo,而检查点之后只有LSN 08是T2的,所以需要redo T2的LSN 08
	\end{enumerate}\textbf{}
	
	\textbf{T1不需要操作的原因}
	
	\begin{enumerate}[noitemsep, label={{\arabic*})}]
		\item T1 在检查点之前已提交(LSN 06),且检查点(LSN 07)会确保其修改已持久化到磁盘。
		\item 恢复时无需对已提交且检查点确认的事务执行任何操作(既不需要 Redo,也不需要 Undo)
	\end{enumerate}\textbf{}
	
	
	\section*{4 死锁检测与处理 (Deadlock Detection)}
	
	\textbf{背景知识}: 当多个事务循环等待对方持有的锁时,就会发生死锁 。数据库管理系统可以通过维护一个等待图 (Waits-for Graph) 来检测死锁 。如果等待图中存在一个环,则表示系统存在死锁 。
	
	\textbf{问题}: 考虑以下两个事务 T3 和 T4 的执行片段:
	
	\begin{table}[!ht]
		\centering
		\begin{tabular}{|l|l|l|}
			\hline
			时间 & T3 & T4  \\ \hline
			1 & lock-X(B) &   \\ \hline
			2 & ... &   \\ \hline
			3 & ... & lock-S(A)  \\ \hline
			4 & ... & ...  \\ \hline
			5 & ... & lock-S(B)  \\ \hline
			6 & lock-X(A) & ...  \\ \hline
		\end{tabular}
	\end{table}
	
	\textbf{要求}:
	
	\begin{enumerate}[noitemsep, label={{\arabic*})}]
		\item 画出在时间点 6 时,系统的事务等待图 (Waits-for Graph)。
		\item 判断此时系统是否处于死锁状态,并解释原因。
		\item 如果存在死锁,请提出一种解除该死锁的方法 。
	\end{enumerate}\textbf{}
	
	\section*{4 解答}
	
	\begin{table}[!ht]
		\centering
		\begin{tabular}{|l|l|l|l|}
			\hline
			事务 & 已持有的锁 & 正在请求的锁 & 阻塞原因  \\ \hline
			T3 & X(B) & X(A) & T4 持有 S(A),不兼容  \\ \hline
			T4 & S(A) & S(B) & T3 持有 X(B),不兼容  \\ \hline
		\end{tabular}
	\end{table}
	
	\begin{tikzpicture}[
		node distance=3cm,
		transaction/.style={circle, draw, minimum size=1cm, font=\large}
		]
		% 定义节点(事务)
		\node[transaction] (T3) {T3};
		\node[transaction, right of=T1] (T4) {T4};
		
		% 绘制双向边(表示环)
		\draw[-{Stealth[scale=1.2]}, thick] (T3) to [bend left=20] node[above] {\small \texttt{}} (T4);
		\draw[-{Stealth[scale=1.2]}, thick] (T4) to [bend left=20] node[below] {\small \texttt{}} (T3);
		
		% 添加标题
		\node[above=0.5cm of T1, font=\bfseries] {等待图};
	\end{tikzpicture}
	
	等待图中存在环(T3 ⇄ T4),表示事务互相等待对方释放锁,形成循环依赖,也就是死锁。
	
	\textbf{解决办法}:
	
	\begin{enumerate}[noitemsep, label={{\arabic*})}]
		\item 选择牺牲者(Victim):
			\begin{enumerate}[noitemsep, label={· }]
				\item 根据优先级、已执行时间或修改量选择一个事务中止(如回滚 T3 或 T4)。
				\item 例如:中止 T4(因其可能持有较少的锁或执行时间较短)。
			\end{enumerate}\textbf{}
		\item 回滚牺牲者:
			\begin{enumerate}[noitemsep, label={· }]
				\item 释放牺牲者持有的所有锁(本例中若中止 T4,则释放 S(A))。
				\item T3 随后可获取 X(A),继续执行。
			\end{enumerate}\textbf{}
		\item 重启事务:
			\begin{enumerate}[noitemsep, label={· }]
				\item 牺牲者(T4)稍后重新启动执行。
			\end{enumerate}\textbf{}
	\end{enumerate}\textbf{}
	
\end{document}
