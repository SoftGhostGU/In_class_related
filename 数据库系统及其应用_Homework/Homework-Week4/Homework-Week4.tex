\documentclass{article}

% 导入宏包
\usepackage{fancyhdr}
\usepackage{ctex}
\usepackage{listings}
\usepackage{graphicx}
\usepackage[a4paper, body={18cm,22cm}]{geometry}
\usepackage{amsmath,amsthm,amssymb,amstext,wasysym,enumerate,graphicx}
\usepackage{float,abstract,booktabs,indentfirst,amsmath}
\usepackage{array}
\usepackage{multirow}
\usepackage{url}
\usepackage{diagbox}
\usepackage{enumitem}
\usepackage{xcolor}
\usepackage{makecell}
\usepackage{tikz}
\usepackage{tcolorbox}
\usetikzlibrary{positioning, arrows.meta}
\usepackage[bookmarks=true, colorlinks, citecolor=blue, linkcolor=black]{hyperref}


% 设置段落
\renewcommand\arraystretch{1.4}
\setlength{\parindent}{2em}
\setCJKmonofont{黑体}

% 设置高亮文字
\newtcbox{\mybox}[1][red]
{on line, arc = 0pt, outer arc = 0pt,
	colback = #1!10!white, colframe = #1!50!black,
	boxsep = 0pt, left = 1pt, right = 1pt, top = 2pt, bottom = 2pt,
	boxrule = 0pt, bottomrule = 1pt, toprule = 1pt}

% 配置代码显示
\lstset{
	xleftmargin = 3em,
	xrightmargin = 3em,
	aboveskip = 1em,
	backgroundcolor = \color{white},
	basicstyle = \small\ttfamily,
	rulesepcolor = \color{gray},
	breaklines = true,
	numbers = left,
	numberstyle = \small,
	numbersep = -14pt,
	keywordstyle = \color{purple}\bfseries,
	commentstyle = \color{green!60!black}, % 修改注释颜色
	stringstyle = \color{red!60!green!90!blue!90},
	morekeywords = {ASSERT, int64_t, uint32_t},
	moreemph = {ASSERT, NULL},
	emphstyle = \color{red}\bfseries,
	moreemph = [2]{int64\_t, uint32\_t, tid\_t, uint8\_t, int16\_t, uint16\_t, int32\_t, size\_t, bool},
	emphstyle = [2]\color{purple}\bfseries,
	frame = shadowbox,
	showspaces = false,
	columns = fixed
	morecomment = [l][\color{green!60!black}]{+}, % 设置以+开头的代码行为绿色
}

%--------------------页眉--------------------%

\pagestyle{fancy}
\fancyhead[L]{}
\fancyhead[R]{}
\fancyhead[C]{华东师范大学软件工程学院作业}
\fancyfoot[C]{-\thepage-}
\renewcommand{\headrulewidth}{1.5pt}

%--------------------标题--------------------%

\begin{document}
	
	\begin{center}
		{\Large{\textbf{\heiti 软件工程学院数据库系统及其应用作业}}}
		\begin{table}[htb]
			\flushleft
			\begin{tabular}{p{0.4\linewidth}p{0.27\linewidth}p{0.28\linewidth}}\\
				\textbf{实验课程}:数据库系统及其应用  & \textbf{年级}:2023级       & \textbf{姓名}:顾翌炜  \\
				\textbf{作业编号}:Week-3    & \textbf{学号}:10235101527 & \textbf{作业日期}:2025/03/05  \\
			\end{tabular}
		\end{table}
	\end{center}
	\rule{\textwidth}{2pt}
	
	\setlength{\parindent}{2em}
	
	\section*{4.12}
	
	假设一个用户想授予另一个用户一个关系上的选择( select ) 访问权限。为什么该用户应该在授权( grant ) 语句中包含(或不包含)granted by current role 子句?
	
	\section*{4.12 解答}
	
	\begin{enumerate}
		\item \textbf{包含 \texttt{GRANTED BY CURRENT ROLE} 子句:}
		\begin{itemize}
			\item 适用于权限通过角色进行管理的场景。如果当前用户是通过某个角色获得的访问权限,并希望通过该角色将权限授予其他用户,可以使用 \texttt{GRANTED BY CURRENT ROLE} 来明确指定。
			\item 这种方式有助于跟踪权限的来源,当权限需要撤销时,可以更容易地撤销特定角色下的授权。
			\item 在某些数据库系统中,权限通过角色传递,使用该子句可以确保权限按照角色继承规则顺利进行授予。
		\end{itemize}
		
		\item \textbf{不包含 \texttt{GRANTED BY CURRENT ROLE} 子句:}
		\begin{itemize}
			\item 如果当前用户的权限是直接授予的,而不是通过角色继承式获得的,那么可以省略该子句。权限将直接授予目标用户,而不是通过角色。
			\item 对于直接授权,直接体现权限来源,不需要引入角色的概念来管理权限分配。
			\item 对于那些不支持通过角色传递权限的数据库系统,该子句是无意义的,因此不需要包含。
		\end{itemize}
	\end{enumerate}
	
	\section*{4.15}
	
	请重写查询:
	
	\begin{verbatim}
			select *
			from section natural join classroom
	\end{verbatim}
	
	不使用自然连接,而是使用具有using 条件的内连接。
	
	\section*{4.15 解答}
	
	\begin{lstlisting}[language=sql, title=4.15 解答, tabsize=2]
		SELECT *
		FROM section
		JOIN classroom
		USING (building, room number);
	\end{lstlisting}
	
	\section*{4.18}
	
	对于图4-12中的数据库,请编写一个查询来找出没有经理的每位员工的ID。请注意.一位员
	工可能只是没有列出经理、或者可能经理值为空。请使用外连接来编写你的查询,然后根本不
	使用外连接再重写查询。
	
	\begin{table}[h!]
		\centering
		\begin{tabular}{|l|}
			\hline
			\textbf{employee (\underline{\textit{ID}}, person\_name, street, city)} \\
			\textbf{works (\underline{\textit{ID}}, company\_name, salary)} \\ 
			\textbf{company (\underline{\textit{company\_name}}, city)} \\ 
			\textbf{manages (\underline{\textit{ID}}, manager\_id)} \\ \hline
		\end{tabular}
		\caption{关系模式示例}
	\end{table}
	
	\section*{4.18 解答}
	
	\begin{enumerate}[noitemsep, label={{\arabic*})}]
		\item 使用外连接
		
		\begin{lstlisting}[language=sql, title=4.18-1 解答, tabsize=2]
		    SELECT employee.ID
		    FROM employee
		    LEFT OUTER JOIN manages
		    ON employee.ID = managers.ID
		    where manages.manager_id IS NULL;
		\end{lstlisting}
		
		\item 不使用外连接
		
		\begin{lstlisting}[language=sql, title=4.18-1 解答, tabsize=2]
		    (SELECT employee.ID
		    FROM employee)
		    except
		    (SELECT managers.ID
		    FROM managers);
		\end{lstlisting}
	\end{enumerate}\textbf{}
	
\end{document}
