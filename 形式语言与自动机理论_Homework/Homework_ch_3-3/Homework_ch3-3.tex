\documentclass{article}

% 导入宏包
\usepackage{fancyhdr}
\usepackage{ctex}
\usepackage{listings}
\usepackage{graphicx}
\usepackage[a4paper, body={18cm,22cm}]{geometry}
\usepackage{amsmath,amsthm,amssymb,amstext,wasysym,enumerate,graphicx}
\usepackage{float,abstract,booktabs,indentfirst,amsmath}
\usepackage{array}
\usepackage{multirow}
\usepackage{url}
\usepackage{diagbox}
\usepackage{enumitem}
\usepackage{xcolor}
\usepackage{makecell}
\usepackage{tikz}
\usepackage{tcolorbox}
\usetikzlibrary{positioning, arrows.meta}
\usepackage[bookmarks=true, colorlinks, citecolor=blue, linkcolor=black]{hyperref}


% 设置段落
\renewcommand\arraystretch{1.4}
\setlength{\parindent}{2em}
\setCJKmonofont{黑体}

% 设置高亮文字
\newtcbox{\mybox}[1][red]
{on line, arc = 0pt, outer arc = 0pt,
	colback = #1!10!white, colframe = #1!50!black,
	boxsep = 0pt, left = 1pt, right = 1pt, top = 2pt, bottom = 2pt,
	boxrule = 0pt, bottomrule = 1pt, toprule = 1pt}

% 配置代码显示
\lstset{
	xleftmargin = 3em,
	xrightmargin = 3em,
	aboveskip = 1em,
	backgroundcolor = \color{white},
	basicstyle = \small\ttfamily,
	rulesepcolor = \color{gray},
	breaklines = true,
	numbers = left,
	numberstyle = \small,
	numbersep = -14pt,
	keywordstyle = \color{purple}\bfseries,
	commentstyle = \color{green!60!black}, % 修改注释颜色
	stringstyle = \color{red!60!green!90!blue!90},
	morekeywords = {ASSERT, int64_t, uint32_t},
	moreemph = {ASSERT, NULL},
	emphstyle = \color{red}\bfseries,
	moreemph = [2]{int64\_t, uint32\_t, tid\_t, uint8\_t, int16\_t, uint16\_t, int32\_t, size\_t, bool},
	emphstyle = [2]\color{purple}\bfseries,
	frame = shadowbox,
	showspaces = false,
	columns = fixed
	morecomment = [l][\color{green!60!black}]{+}, % 设置以+开头的代码行为绿色
}

%--------------------页眉--------------------%

\pagestyle{fancy}
\fancyhead[L]{}
\fancyhead[R]{}
\fancyhead[C]{华东师范大学软件工程学院作业}
\fancyfoot[C]{-\thepage-}
\renewcommand{\headrulewidth}{1.5pt}

%--------------------标题--------------------%

\begin{document}
	
	\begin{center}
		{\Large{\textbf{\heiti 软件工程学院形式语言与自动机理论作业}}}
		\begin{table}[htb]
			\flushleft
			\begin{tabular}{p{0.4\linewidth}p{0.27\linewidth}p{0.28\linewidth}}\\
				\textbf{实验课程}:形式语言与自动机理论  & \textbf{年级}:2023级       & \textbf{姓名}:顾翌炜  \\
				\textbf{实验编号}:ch-3-3    & \textbf{学号}:10235101527 & \textbf{作业日期}:2025/04/30  \\
			\end{tabular}
		\end{table}
	\end{center}
	\rule{\textwidth}{2pt}
	
	\section*{课后作业}
	
	\begin{enumerate}[noitemsep, label={{\arabic*})}]
		\item 将如下下推自动机 $P = (\{p, q\}, \{0, 1\}, \{X, Z\}, \delta, p, Z)$
		转化为一个等价的上下文无关文法。
		\item 将该文法化简。
		\item 将该文法转化为Chomsky范式形式。
	\end{enumerate}\textbf{}
	
	\begin{table}[htbp]
		\centering
		\caption{下推自动机的转移函数表}
		\label{tab:transition}
		\begin{tabular}{|c|c|c|c|c|c|c|}
			\hline
			& $0, Z$ & $0, X$ & $1, Z$ & $1, X$ & $\epsilon, Z$ & $\epsilon, X$ \\
			\hline
			$\to p$ & $--$ & $q, X$ & $p, XZ$ & $p, XX$ & $p, \epsilon$ & $--$ \\
			\hline
			$q$ & $p, Z$ & $--$ & $--$ & $q, \epsilon$ & $--$ & $--$ \\
			\hline
		\end{tabular}
	\end{table}
	
	\section*{解答 1}
	
	\begin{itemize}
		\item 起始符号定义:
		$ S \rightarrow [pZp] \mid [pZq] $
		
		\item 根据转移函数 $\delta(p, 0, X) = \{(q, X)\}$:
		\begin{itemize}
			\item $[pXp] \rightarrow 0[qXp]$
			\item $[pXq] \rightarrow 0[qXq]$
		\end{itemize}
		
		\item 根据转移函数 $\delta(p, 1, Z) = \{(p, XZ)\}$:
		\begin{itemize}
			\item $[pZp] \rightarrow 1[pXp][pZp] \mid 1[pXq][qZp]$
			\item $[pZq] \rightarrow 1[pXp][pZq] \mid 1[pXq][qZq]$
		\end{itemize}
		
		\item 根据转移函数 $\delta(p, 1, X) = \{(p, XX)\}$:
		\begin{itemize}
			\item $[pXp] \rightarrow 1[pXp][pXp] \mid 1[pXq][qXp]$
			\item $[pXq] \rightarrow 1[pXp][pXq] \mid 1[pXq][qXq]$
		\end{itemize}
		
		\item 根据转移函数 $\delta(p, \epsilon, Z) = \{(p, \epsilon)\}$:
		\begin{itemize}
			\item $[pZp] \rightarrow \epsilon$
		\end{itemize}
		
		\item 根据 $\delta(q, 0, Z) = \{(p, Z)\}$:
		\begin{itemize}
			\item $[qZp] \rightarrow 0[pZp]$
			\item $[qZq] \rightarrow 0[pZq]$
		\end{itemize}
		
		\item 根据 $\delta(q, 1, X) = \{(q, \epsilon)\}$:
		\begin{itemize}
			\item $[qXq] \rightarrow 1$
		\end{itemize}
	\end{itemize}
	
	将其中的内容都用ABCD等来表示,可以得到:
	
	\begin{itemize}
		\item 起始符号:
		$ S \rightarrow A \mid B $
		
		\item 产生式规则:
		\begin{itemize}
			\item \( A \rightarrow 1CA \mid 1DE \mid \epsilon \)
			\item \( B \rightarrow 1CB \mid 1DF \)
			\item \( C \rightarrow 0G \mid 1CC \mid 1DG \)
			\item \( D \rightarrow 0H \mid 1CD \mid 1DH \)
			\item \( E \rightarrow 0A \)
			\item \( F \rightarrow 0E \)
			\item \( H \rightarrow 1 \)
		\end{itemize}
	\end{itemize}
	
	\section*{解答 2}
	
	\begin{enumerate}[noitemsep, label={{\arabic*})}]
		\item 消除 $\epsilon$ 产生式,删除 $ A \to \epsilon $,得到结果:
		
		\begin{align*}
			S &\rightarrow A \mid B \\
			A &\rightarrow 1CA \mid 1DE \mid 1C \\
			B &\rightarrow 1CB \mid 1DF \\
			C &\rightarrow 0G \mid 1CC \mid 1DG \\
			D &\rightarrow 0H \mid 1CD \mid 1DH \\
			E &\rightarrow 0A \mid 0 \\
			F &\rightarrow 0E \\
			H &\rightarrow 1
		\end{align*}
		
		\item 消除单位产生式,消除 $ S \to A, S \to B$,得到结果:
		
		\begin{align*}
			S &\rightarrow 1CA \mid 1DE \mid 1C \mid 1CB \mid 1DF \\
			A &\rightarrow 1CA \mid 1DE \mid 1C \\
			B &\rightarrow 1CB \mid 1DF \\
			C &\rightarrow 0G \mid 1CC \mid 1DG \\
			D &\rightarrow 0H \mid 1CD \mid 1DH \\
			E &\rightarrow 0A \mid 0 \\
			F &\rightarrow 0E \\
			H &\rightarrow 1
		\end{align*}
		
		\item 消除无用符号
		
		\begin{enumerate}
			\item 计算产生符号:$g(G) = \{0, 1, E, F, H, D, B, A, S\}$,去除符号$C, G$,得到
			\begin{align*}
				S &\rightarrow 1DE \mid 1DF \\
				A &\rightarrow 1DE \quad B \rightarrow 1DF \\
				D &\rightarrow 0H \mid 1DH \\
				E &\rightarrow 0A \mid 0 \quad F \rightarrow 0E \quad H \rightarrow 1
			\end{align*}
			
			\item 计算到达符号:$r(G) = \{S, D, E, F, H, A\}$,去除符号$B$,得到
			\begin{align*}
				S &\rightarrow 1DE \mid 1DF \\
				A &\rightarrow 1DE \quad D \rightarrow 0H \mid 1DH \\
				E &\rightarrow 0A \mid 0 \quad F \rightarrow 0E \quad H \rightarrow 1
			\end{align*}
		\end{enumerate}
		
	\end{enumerate}\textbf{}
	
	\section*{解答 3}
	
	\begin{enumerate}
		\item 替换所有长度大于1的终止符串,得到 \(E \rightarrow 0, H \rightarrow 1\)
		\begin{align*}
			S &\rightarrow HDE \mid HDF \\
			A &\rightarrow HDE \\
			D &\rightarrow EH \mid HDH \\
			E &\rightarrow EA \mid 0 \quad F \rightarrow EE \quad H \rightarrow 1
		\end{align*}
		
		\item 拆分长度大于2的非终止符串,得到
		\begin{align*}
			S &\rightarrow HC_1 \mid HC_2 \quad C_1 \rightarrow DE \quad C_2 \rightarrow DF \\
			A &\rightarrow HC_1 \quad D \rightarrow EH \mid HC_3 \quad C_3 \rightarrow DH \\
			E &\rightarrow EA \mid 0 \quad F \rightarrow EE \quad H \rightarrow 1
		\end{align*}
	\end{enumerate}
	
\end{document}
