\documentclass{article}

% 导入宏包
\usepackage{fancyhdr}
\usepackage{ctex}
\usepackage{listings}
\usepackage{graphicx}
\usepackage[a4paper, body={18cm,22cm}]{geometry}
\usepackage{amsmath,amsthm,amssymb,amstext,wasysym,enumerate,graphicx}
\usepackage{float,abstract,booktabs,indentfirst,amsmath}
\usepackage{array}
\usepackage{multirow}
\usepackage{url}
\usepackage{diagbox}
\usepackage{enumitem}
\usepackage{xcolor}
\usepackage{makecell}
\usepackage{tikz}
\usepackage{tcolorbox}
\usetikzlibrary{positioning, arrows.meta}
\usepackage[bookmarks=true, colorlinks, citecolor=blue, linkcolor=black]{hyperref}


% 设置段落
\renewcommand\arraystretch{1.4}
\setlength{\parindent}{2em}
\setCJKmonofont{黑体}

% 设置高亮文字
\newtcbox{\mybox}[1][red]
{on line, arc = 0pt, outer arc = 0pt,
	colback = #1!10!white, colframe = #1!50!black,
	boxsep = 0pt, left = 1pt, right = 1pt, top = 2pt, bottom = 2pt,
	boxrule = 0pt, bottomrule = 1pt, toprule = 1pt}

% 配置代码显示
\lstset{
	xleftmargin = 3em,
	xrightmargin = 3em,
	aboveskip = 1em,
	backgroundcolor = \color{white},
	basicstyle = \small\ttfamily,
	rulesepcolor = \color{gray},
	breaklines = true,
	numbers = left,
	numberstyle = \small,
	numbersep = -14pt,
	keywordstyle = \color{purple}\bfseries,
	commentstyle = \color{green!60!black}, % 修改注释颜色
	stringstyle = \color{red!60!green!90!blue!90},
	morekeywords = {ASSERT, int64_t, uint32_t},
	moreemph = {ASSERT, NULL},
	emphstyle = \color{red}\bfseries,
	moreemph = [2]{int64\_t, uint32\_t, tid\_t, uint8\_t, int16\_t, uint16\_t, int32\_t, size\_t, bool},
	emphstyle = [2]\color{purple}\bfseries,
	frame = shadowbox,
	showspaces = false,
	columns = fixed
	morecomment = [l][\color{green!60!black}]{+}, % 设置以+开头的代码行为绿色
}

%--------------------页眉--------------------%

\pagestyle{fancy}
\fancyhead[L]{}
\fancyhead[R]{}
\fancyhead[C]{华东师范大学软件工程学院作业}
\fancyfoot[C]{-\thepage-}
\renewcommand{\headrulewidth}{1.5pt}

%--------------------标题--------------------%

\begin{document}
	
	\begin{center}
		{\Large{\textbf{\heiti 软件工程学院形式语言与自动机理论作业}}}
		\begin{table}[htb]
			\flushleft
			\begin{tabular}{p{0.4\linewidth}p{0.27\linewidth}p{0.28\linewidth}}\\
				\textbf{实验课程}:形式语言与自动机理论  & \textbf{年级}:2023级       & \textbf{姓名}:顾翌炜  \\
				\textbf{实验编号}:ch-2-1    & \textbf{学号}:10235101527 & \textbf{作业日期}:2025/03/04  \\
			\end{tabular}
		\end{table}
	\end{center}
	\rule{\textwidth}{2pt}
	
	\section*{课后作业}
	
	\begin{enumerate}
		\item 给出语言 $\{a^{2n}b^m \mid n \geq 1, m \geq 1\}$ 的一个正则文法(3型文法),并在以下示例中演示其最左推导过程:
		\begin{itemize}
			\item $aaaab$
			\item $aabbb$
		\end{itemize}
		\item 给出一个上下文无关文法(2型文法)描述如下语言,要求其中 0 和 1 的比例为 2 比 1,并在以下示例中演示其最右推导过程:
		\begin{itemize}
			\item $001100$
			\item $11000100$
		\end{itemize}
	\end{enumerate}
	
	\section*{解答}
	
	\begin{enumerate}[noitemsep, label={{\arabic*}.}]
		\item  第一题:正则文法
		
		\begin{tcolorbox}[colback = blue!25!white, colframe = blue!75!black]
			$ S \rightarrow aA $ \\
			$ A \rightarrow aB $ \\
			$ B \rightarrow aA \mid Bb \mid \epsilon $
		\end{tcolorbox}
		
		最左推导过程:
		
		\begin{itemize}
			\item $aaaab$:  
			\setlength{\jot}{0pt}
			\begin{align*}
				S &\rightarrow aA      &A \to aB\\
				  &\rightarrow aaB     &B \to aA\\
				  &\rightarrow aaaA    &A \to aB\\
				  &\rightarrow aaaaB   &B \to Bb\\
				  &\rightarrow aaaaBb  &B \to \epsilon\\
				  &\rightarrow aaaa b  &
			\end{align*}
			
			\item $aabbb$: 
			\setlength{\jot}{0pt}
			\begin{align*}
				S &\rightarrow aA      &A \to aB\\
				  &\rightarrow aaB     &B \to Bb\\
				  &\rightarrow aaBb    &A \to Bb\\
				  &\rightarrow aaBbb   &B \to Bb\\
				  &\rightarrow aaaBbbb &B \to \epsilon\\
				  &\rightarrow aaa bbb &
			\end{align*}
		\end{itemize}
		
		\item 第二题:2型文法
		
		\begin{tcolorbox}[colback = blue!25!white, colframe = blue!75!black]
			\begin{align*}
				S \rightarrow & 001S \mid 010S \mid 100S \mid \\
				              & 00S1 \mid 01S0 \mid 10S0 \mid \\
				              & 0S01 \mid 0S10 \mid 1S00 \mid \\
				              & S001 \mid S010 \mid S100 \mid \epsilon
			\end{align*}
		\end{tcolorbox}
		
		最右推导过程:
		
		\begin{itemize}
			\item $001100$:  
			\setlength{\jot}{0pt}
			\begin{align*}
				S &\rightarrow S100    &S \to S001 \\
				  &\rightarrow S001100 &B \to \epsilon \\
				  &\rightarrow 001100  
			\end{align*}
			
			\item $110000100$: 
			\setlength{\jot}{0pt}
			\begin{align*}
				S &\rightarrow S100        &S \to 1S00\\
				  &\rightarrow 1S00100     &S \to S100\\
				  &\rightarrow 1S10000100  &S \to \epsilon\\
				  &\rightarrow 110000100 
			\end{align*}
		\end{itemize}
		
	\end{enumerate}\textbf{}
	
\end{document}
