\documentclass{article}

% 导入宏包
\usepackage{fancyhdr}
\usepackage{ctex}
\usepackage{listings}
\usepackage{graphicx}
\usepackage[a4paper, body={18cm,22cm}]{geometry}
\usepackage{amsmath,amsthm,amssymb,amstext,wasysym,enumerate,graphicx}
\usepackage{float,abstract,booktabs,indentfirst,amsmath}
\usepackage{array}
\usepackage{multirow}
\usepackage{url}
\usepackage{diagbox}
\usepackage{enumitem}
\usepackage{xcolor}
\usepackage{makecell}
\usepackage{tikz}
\usepackage{tcolorbox}
\usetikzlibrary{positioning, arrows.meta}
\usepackage[bookmarks=true, colorlinks, citecolor=blue, linkcolor=black]{hyperref}


% 设置段落
\renewcommand\arraystretch{1.4}
\setlength{\parindent}{2em}
\setCJKmonofont{黑体}

% 设置高亮文字
\newtcbox{\mybox}[1][red]
{on line, arc = 0pt, outer arc = 0pt,
	colback = #1!10!white, colframe = #1!50!black,
	boxsep = 0pt, left = 1pt, right = 1pt, top = 2pt, bottom = 2pt,
	boxrule = 0pt, bottomrule = 1pt, toprule = 1pt}

% 配置代码显示
\lstset{
	xleftmargin = 3em,
	xrightmargin = 3em,
	aboveskip = 1em,
	backgroundcolor = \color{white},
	basicstyle = \small\ttfamily,
	rulesepcolor = \color{gray},
	breaklines = true,
	numbers = left,
	numberstyle = \small,
	numbersep = -14pt,
	keywordstyle = \color{purple}\bfseries,
	commentstyle = \color{green!60!black}, % 修改注释颜色
	stringstyle = \color{red!60!green!90!blue!90},
	morekeywords = {ASSERT, int64_t, uint32_t},
	moreemph = {ASSERT, NULL},
	emphstyle = \color{red}\bfseries,
	moreemph = [2]{int64\_t, uint32\_t, tid\_t, uint8\_t, int16\_t, uint16\_t, int32\_t, size\_t, bool},
	emphstyle = [2]\color{purple}\bfseries,
	frame = shadowbox,
	showspaces = false,
	columns = fixed
	morecomment = [l][\color{green!60!black}]{+}, % 设置以+开头的代码行为绿色
}

%--------------------页眉--------------------%

\pagestyle{fancy}
\fancyhead[L]{}
\fancyhead[R]{}
\fancyhead[C]{华东师范大学软件工程学院作业}
\fancyfoot[C]{-\thepage-}
\renewcommand{\headrulewidth}{1.5pt}

%--------------------标题--------------------%

\begin{document}
	
	\begin{center}
		{\Large{\textbf{\heiti 软件工程学院形式语言与自动机理论作业}}}
		\begin{table}[htb]
			\flushleft
			\begin{tabular}{p{0.4\linewidth}p{0.27\linewidth}p{0.28\linewidth}}\\
				\textbf{实验课程}:形式语言与自动机理论  & \textbf{年级}:2023级       & \textbf{姓名}:顾翌炜  \\
				\textbf{实验编号}:ch-1    & \textbf{学号}:10235101527 & \textbf{作业日期}:2025/02/26  \\
			\end{tabular}
		\end{table}
	\end{center}
	\rule{\textwidth}{2pt}
	
	\section*{课后作业}
	
	\begin{itemize}
		\item \textbf{Fibonacci numbers}
		\begin{itemize}
			\item basis: $F_0 = 0, F_1 = 1$
			\item recursion: if $F_i = m, F_{i+1} = n$ then $F_{i+2} = m + n$ for $i \geq 0$
		\end{itemize}
		\item 使用数学归纳法证明: \\
		$F_k = \frac{\varphi^k - \psi^k}{\varphi - \psi}$ where $\varphi = \frac{1 + \sqrt{5}}{2}$, $\psi = \frac{1 - \sqrt{5}}{2}$
	\end{itemize}
	
	\section*{解答}
	
	\begin{enumerate}
		\item \textbf{基础步骤:} 
		
		首先证明公式对于 $k=0$ 和 $k=1$ 成立。
		
		对于 $k=0$: 
		$
		F_0 = \frac{\phi^0 - \psi^0}{\phi - \psi} = \frac{1 - 1}{\phi - \psi} = 0
		$
		
		对于 $k=1$: 
		$
		F_1 = \frac{\phi^1 - \psi^1}{\phi - \psi} = \frac{\phi - \psi}{\phi - \psi} = 1
		$
		
		\item \textbf{归纳步骤:} 
		
		假设对 $k=n$ 成立, 即: 
		$
		F_n = \frac{\phi^n - \psi^n}{\phi - \psi}
		$
		
		假设对 $k=n+1$ 也成立, 即: 
		$
		F_{n+1} = \frac{\phi^{n+1} - \psi^{n+1}}{\phi - \psi}
		$
		
		\item \textbf{证明对 $k=n+2$ 成立:}
		
		根据递推公式:
		$
		F_{n+2} = F_{n+1} + F_n
		$
		
		可以得到:
		
		$
		F_{n+2} = \left( \frac{\phi^{n+1} - \psi^{n+1}}{\phi - \psi} \right) + \left( \frac{\phi^n - \psi^n}{\phi - \psi} \right)
		$
		
		$
		F_{n+2} = \frac{\phi^{n+1} + \phi^n - (\psi^{n+1} + \psi^n)}{\phi - \psi}
		$
		
		由于 $\phi$ 和 $\psi$ 满足:$\varphi = \frac{1 + \sqrt{5}}{2}$, $\psi = \frac{1 - \sqrt{5}}{2}$,
		
		而这两个数字有以下公式: $\phi^2 = \phi + 1$ 、 $\psi^2 = \psi + 1$, 
		
		则可以得到:
		$
		\phi^{n+2} = \phi^{n+1} + \phi^n \quad $ 、 $ \quad \psi^{n+2} = \psi^{n+1} + \psi^n
		$
		
		从而可以得到:
		$
		F_{n+2} = \frac{\phi^{n+2} - \psi^{n+2}}{\phi - \psi}
		$
		
		根据数学归纳法, 题目中的公式
		
		$$
		F_k = \frac{\varphi^k - \psi^k}{\varphi - \psi} \ (\ \varphi = \frac{1 + \sqrt{5}}{2},\ \psi = \frac{1 - \sqrt{5}}{2}\ )
		$$
		
		在 $k = n$ 和 $k = n+1$ 成立的时候,对于 $k = n+2$ 也成立,即对于任意非负整数均成立
		
		故得到结论:原式对于所有的 $k \geq 0$ 成立。
	\end{enumerate}
	
\end{document}
